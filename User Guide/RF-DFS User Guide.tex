\documentclass[titlepage]{article}

% Suppresses chktex warnings 
% chktex-file 1
% chktex-file 3
% chktex-file 8
% chktex-file 10
% chktex-file 17
% chktex-file 24
% chktex-file 26
% chktex-file 36
% chktex-file 44
% chktex-file 46

\usepackage{geometry}
\usepackage{tikz}
\usetikzlibrary{fit}
\usetikzlibrary{intersections}
\usepackage{pgf}        % Used to import pgf figures with \input
\def\mathdefault#1{#1}  % Fixes 'Undefined control sequence \mathdefault' in imported pgf files
\usepackage{pgfplots}
\usepgfplotslibrary{fillbetween}
\pgfplotsset{compat=1.18}
\pgfplotsset{hide xscale/.style={/pgfplots/xtick scale label code/.code={}}}
\pgfplotsset{hide yscale/.style={/pgfplots/ytick scale label code/.code={}}}
\usepackage{amsmath}
\usepackage{upgreek}
\usepackage{url}
\hyphenation{}	        % Correct bad hyphenation
\usepackage{graphicx}   % For including figures and pictures
\usepackage{float}      % Used to fix location of images i.e.\begin{figure}[H]
\usepackage{fancyhdr}   % For headers and footers
\pagestyle{fancy}
\usepackage{lastpage}   % Allows referencing last page (for footer)
\usepackage{titling}    % To reference title, author, and date
\usepackage{array}      % For fixed-width tables
\usepackage{makecell}   % Formats individual table cells
\usepackage{multirow}   % For multi-row columns (Top left justify header)
\usepackage[american]{circuitikz} % Used to draw circuit and block diagrams
\usepackage{appendix}   
\usepackage{colortbl}   % For coloring table cells
\definecolor{nraoblue}{rgb}{0.776,0.851,0.945}  % Color of table headers
\usepackage{enumitem}   % For formatting enumerated lists
\usepackage[parfill]{parskip}               % Removes paragraph indentation, adds line break
\usepackage[hang, flushmargin]{footmisc}    % Removes footnote indentation
\setenumerate[1]{label={(\arabic*)}}
\usepackage[backend=biber,style=numeric,sorting=none]{biblatex}
\addbibresource{references.bib}
\usepackage{hyperref}   % Allows hyperlinks
\hypersetup{
  colorlinks   = true,  % Colors links instead of ugly boxes
  filecolor    = blue,  % Color for local hyperlinks
  linkcolor    = ., % Color for internal hyperlinks
  citecolor    = ., % Color for citations
  urlcolor     = blue,  % Color for linked urls
}
\usepackage{caption}    % Changes figure/table caption color and size
\definecolor{captioncolor}{RGB}{79,129,189}
\captionsetup{font={
    small, 
    color=captioncolor,
    bf,
}}
 

% Changes font from Computer Modern to Gill Sans MT
% This requires XeLaTeX or LuaTeX to function
% Add "latex-workshop.latex.recipe.default": "latexmk (xelatex)" to workspace.json
\usepackage{fontspec}
\setmainfont{gil.ttf}[
    BoldFont        = gil-b.ttf,
    ItalicFont      = gil-i.ttf,
    BoldItalicFont  = gil-bi.ttf,
]

% Global TikZ parameters
\tikzset{every picture/.style={/utils/exec={\fontfamily{lmr}}}}
\ctikzset{sources/fill=red!20}
\ctikzset{chips/fill=cyan!20}
\ctikzset{electromechanicals/fill=blue!20}
\ctikzset{blocks/fill=green!20}
\ctikzset{amplifiers/fill=yellow!30}
\tikzset{amp/.append style={fill=yellow!30}}
\tikzset{twoport/.append style={fill=cyan!20}}
% TikZ node[server] symbol
\makeatletter
\pgfkeys{/pgf/.cd,
  parallelepiped offset x/.initial=2mm,
  parallelepiped offset y/.initial=2mm
}
\pgfdeclareshape{parallelepiped}
{
  \inheritsavedanchors[from=rectangle] % this is nearly a rectangle
  \inheritanchorborder[from=rectangle]
  \inheritanchor[from=rectangle]{north}
  \inheritanchor[from=rectangle]{north west}
  \inheritanchor[from=rectangle]{north east}
  \inheritanchor[from=rectangle]{center}
  \inheritanchor[from=rectangle]{west}
  \inheritanchor[from=rectangle]{east}
  \inheritanchor[from=rectangle]{mid}
  \inheritanchor[from=rectangle]{mid west}
  \inheritanchor[from=rectangle]{mid east}
  \inheritanchor[from=rectangle]{base}
  \inheritanchor[from=rectangle]{base west}
  \inheritanchor[from=rectangle]{base east}
  \inheritanchor[from=rectangle]{south}
  \inheritanchor[from=rectangle]{south west}
  \inheritanchor[from=rectangle]{south east}
  \backgroundpath{
    % store lower right in xa/ya and upper right in xb/yb
    \southwest \pgf@xa=\pgf@x \pgf@ya=\pgf@y
    \northeast \pgf@xb=\pgf@x \pgf@yb=\pgf@y
    \pgfmathsetlength\pgfutil@tempdima{\pgfkeysvalueof{/pgf/parallelepiped
      offset x}}
    \pgfmathsetlength\pgfutil@tempdimb{\pgfkeysvalueof{/pgf/parallelepiped
      offset y}}
    \def\ppd@offset{\pgfpoint{\pgfutil@tempdima}{\pgfutil@tempdimb}}
    \pgfpathmoveto{\pgfqpoint{\pgf@xa}{\pgf@ya}}
    \pgfpathlineto{\pgfqpoint{\pgf@xb}{\pgf@ya}}
    \pgfpathlineto{\pgfqpoint{\pgf@xb}{\pgf@yb}}
    \pgfpathlineto{\pgfqpoint{\pgf@xa}{\pgf@yb}}
    \pgfpathclose
    \pgfpathmoveto{\pgfqpoint{\pgf@xb}{\pgf@ya}}
    \pgfpathlineto{\pgfpointadd{\pgfpoint{\pgf@xb}{\pgf@ya}}{\ppd@offset}}
    \pgfpathlineto{\pgfpointadd{\pgfpoint{\pgf@xb}{\pgf@yb}}{\ppd@offset}}
    \pgfpathlineto{\pgfpointadd{\pgfpoint{\pgf@xa}{\pgf@yb}}{\ppd@offset}}
    \pgfpathlineto{\pgfqpoint{\pgf@xa}{\pgf@yb}}
    \pgfpathmoveto{\pgfqpoint{\pgf@xb}{\pgf@yb}}
    \pgfpathlineto{\pgfpointadd{\pgfpoint{\pgf@xb}{\pgf@yb}}{\ppd@offset}}
  }
}
\makeatother

\tikzset{
  ports/.style={
    line width=0.3pt,
    top color=gray!20,
    bottom color=gray!80
  },
  server/.style={
    parallelepiped,
    fill=white, draw,
    minimum width=0.35cm,
    minimum height=0.75cm,
    parallelepiped offset x=3mm,
    parallelepiped offset y=2mm,
    xscale=-1,
    path picture={
      \draw[top color=gray!5,bottom color=gray!40]
      (path picture bounding box.south west) rectangle 
      (path picture bounding box.north east);
      \coordinate (A-center) at ($(path picture bounding box.center)!0!(path
        picture bounding box.south)$);
      \coordinate (A-west) at ([xshift=-0.575cm]path picture bounding box.west);
      \draw[ports]([yshift=0.1cm]$(A-west)!0!(A-center)$)
        rectangle +(0.2,0.065);
      \draw[ports]([yshift=0.01cm]$(A-west)!0.085!(A-center)$)
        rectangle +(0.15,0.05);
      \fill[black]([yshift=-0.35cm]$(A-west)!-0.1!(A-center)$)
        rectangle +(0.235,0.0175);
      \fill[black]([yshift=-0.385cm]$(A-west)!-0.1!(A-center)$)
        rectangle +(0.235,0.0175);
      \fill[black]([yshift=-0.42cm]$(A-west)!-0.1!(A-center)$)
        rectangle +(0.235,0.0175);
    }  
  },
}

% The following lines define a new command \nraocite that cites references in NRAO format RD0X by redefining \cite command to remove brackets. \nraoprecite includes a prenote.
\DeclareCiteCommand{\cite} [\mkbibemph{\emph{}}]
  {\usebibmacro{prenote}}
  {\usebibmacro{citeindex}%
   \printtext[bibhyperref]{\usebibmacro{cite}}}
  {\multicitedelim}
  {\usebibmacro{postnote}}
\DeclareCiteCommand*{\cite} [\mkbibemph\emph{}]
  {\usebibmacro{prenote}}
  {\usebibmacro{citeindex}
   \printtext[bibhyperref]{\usebibmacro{citeyear}}}
  {\multicitedelim}
  {\usebibmacro{postnote}}
\newcommand{\nraocite}[1]{[RD0\cite{#1}]}
\newcommand{\nraoprecite}[2][]{[RD0\cite{#2}{, #1}]}


% ASSIGN TITLE, AUTHOR, DATE, DOCUMENT NUMBER, STATUS, HERE
\title{Title goes here}
\author{Author
    }% <-this % stops a space
\date{Jan 1, 2000}
\def\docnum{DOC NUM}
\def\status{\textcolor{red}{DRAFT}}

% HEADERS AND FOOTERS
\renewcommand{\headrulewidth}{0pt}
\fancyheadoffset[L,R]{2cm}
\fancyhead[L]{\vspace{-1cm}\includegraphics[width=2cm]{images/NRAO Logo Badge.png}}
\fancyhead[R]{
\renewcommand{\arraystretch}{1.4}
\renewcommand{\arrayrulewidth}{0.25pt}
\begin{tabular}{|w{l}{6.7cm}|w{l}{4.9cm}|w{l}{3.4cm}|}
    \hline
        \multirow[t]{2}{6.7cm}{\textit{\textbf{Title:}} \thetitle} &
        \textit{\textbf{Owner:}} \theauthor &
        \textit{\textbf{Date:}}  \thedate \\
            &   &   \\
    \hline
        \multicolumn{2}{|l|}{
            \textit{\textbf{NRAO Doc \#:}} \docnum  
        } &
        \textit{\textbf{Version:}} A \\
    \hline
\end{tabular}
}
\fancyfoot[C]{Page \textbf{\thepage} of \textbf{\pageref{LastPage}}}

 % DOCUMENT STARTS HERE
\begin{document}
\setlength{\leftmargin}{1in}        % Sets margin
\setlength{\rightmargin}{1in}       % Sets margin
\setlength{\voffset}{-1.2in}        % Moves header up
\setlength{\headheight}{3.5cm}      % Defines header height
\setlength{\textheight}{591pt}      % Extends text/body size down
\setlength{\footskip}{60pt}         % Defines gap between body and footer

\thispagestyle{fancy}
\begin{center}
     \includegraphics[width=5cm]{images/NRAO Logo Badge.png} \\
     \vspace*{0.5cm}
     \textbf{\huge\thetitle} \\
     \vspace*{0.5cm}
     \large\docnum \\
     \Large Status: \status \\
     \vspace*{0.6cm} \large
     % FIRST TABLE HERE
     \begin{tabular}{|m{6.93cm}|m{4.5cm}|m{2cm}|} \hline
        \rowcolor{nraoblue}
        \textbf{Prepared By} & \textbf{Organization} & \textbf{Date} \\ \hline
        \makecell[l]{Author\\Title} & NRAO Electronics Div. & 1/1/2000 \\ 
        \hline
    \end{tabular} \\
    \vspace*{0.8cm}
    \begin{tabular}{|m{3cm}|m{3.5cm}|m{6.93cm}|} \hline
        \rowcolor{nraoblue}
        \textbf{Approvals} & \textbf{Organization} & \textbf{Signatures} \\ \hline
    \end{tabular}
    \renewcommand{\arraystretch}{2}
    % CONTENT OF SECOND TABLE HERE
    \begin{tabular}{|w{l}{3cm}|m{3.5cm}|m{6.93cm}|} \hline
        \parbox{3cm}{\raggedright
        Name\\Title
        } & NRAO Electronics Division \raggedright &  \\ 
        \hline
        \parbox{3cm}{\raggedright
        Name\\Title
        } & NRAO Electronics Division \raggedright &  \\ 
        \hline
        \parbox{3cm}{\raggedright
        Name\\Title
        } & NRAO Electronics Division \raggedright &  \\ 
        \hline
    \end{tabular} \\
    \renewcommand{\arraystretch}{1}
    \vspace*{0.8cm}
    \begin{tabular}{|m{3cm}|m{3.5cm}|m{6.93cm}|} \hline
        \rowcolor{nraoblue}
        \textbf{Released By} & \textbf{Organization} & \textbf{Signature} \\ \hline
    \end{tabular}
    \renewcommand{\arraystretch}{1.5}
    % CONTENT OF THIRD TABLE HERE
    \begin{tabular}{|m{3cm}|m{3.5cm}|m{6.93cm}|} \hline
        \parbox{3cm}{\raggedright
        Name\\Title
        } & NRAO Electronics Division \raggedright &  \\ 
        \hline
    \end{tabular}
    \renewcommand{\arraystretch}{1}
\end{center}

% CHANGE RECORD
\newpage
\section*{Change Record}
\begin{center}
\renewcommand{\arraystretch}{1.2}
    \begin{tabular}{|m{1.5cm}|m{2.2cm}|m{2.5cm}|m{1.7cm}|m{5cm}|} \hline
        \rowcolor{nraoblue}
        Version & Date & Author & Affected\newline Section(s) & Reason\\ \hline
    \end{tabular}
\renewcommand{\arraystretch}{1.6}
    \begin{tabular}{|m{1.5cm}|m{2.2cm}|m{2.5cm}|m{1.7cm}|m{5cm}|} \hline
        01 & Aug 15, 2023 & R. Nguyen & All & Initial Draft \\ \hline
        02 & Aug 16, 2023 & \makecell[l]{T. Anderson\\R. Nguyen} & All & Edits \\ \hline
        A  & Aug 16, 2023 & T. Anderson & All & Review \\ \hline
          &          &           &     &               \\ \hline
          &          &           &     &               \\ \hline
          &          &           &     &               \\ \hline
          &          &           &     &               \\ \hline
    \end{tabular}
\renewcommand{\arraystretch}{1}
\end{center}

\newpage
\tableofcontents
\listoffigures
\thispagestyle{fancy}
\newpage

% DOCUMENT STARTS HERE
\section{Introduction}

\subsection{What is the RF-DFS and RF-EMS?}
copy from readme/wiki/plan or something

The Radio Frequency - Direction Finding System and Environmental Monitoring System are tools utilized by the NRAO-NM Interference Protection Group to identify and locate RF interference at the Very Large Array. The former is a 3-meter dish antenna with azimuth and elevation drives whereas the latter is a 50-foot tower with an omnidirectional club antenna. Both are connected to the RFI shack and controlled by a Python front end. This interface allows a user to control RF switching between both antennas and log data through a spectrum analyzer. 

\subsection{System Overview}
The three locations of note are the RFI shack, the RF-DFS dish, and the RF-EMS tower. A block diagram which shows the breakdown of major components and their connections is shown in Figure~\ref{fig:sysblock2}. The user will be able to control all IO functions through the Python application run on the PC (emsmonpc2).  SCPI commands issued over GPIB allow the user to control and view the spectrum analyzer display on the PC and a serial connection facilitates the same function with the motor controller. This motor controller is a Parker Hannifin ACR9000 which drives 2 Aries AR-04AE servos; one each for azimuth and elevation. RF switching is controlled by a PLC in the RFI shack (not shown).

\begin{figure}[!ht]
  \begin{center}
      \begin{circuitikz}
          % Shack
          \draw(0, 0) node[server, scale=1.5, name=computer]{};
          % \draw(computer.east) ++ (0.5, 0.2) node[anchor=west](computerlabel){PC};
          \draw(computer.west) node[anchor=east]{};
          \ctikzset{bipoles/oscope/waveform=triangle}
          \ctikzset{bipoles/oscope/width=1.0}
          \draw(computer.west)  ++ (-1, 0.2) 
          coordinate(ctrwest)-- ++ (-1.2, 0)
          node[oscopeshape, anchor=east](spec){};
          \draw(spec.west) -- ++ (-1.2, 0)
          node[twoportshape, anchor=east](shackswitch){};
          \draw(shackswitch.center) node[spdt, scale=0.6, xscale=-1]{};
          \draw(ctrwest) ++ (-0.6, 0) node[anchor=south]{GPIB};
          \draw(spec.south) node[anchor=north]{N9040b};
          \draw(spec.in 1) node[ocirc, scale=0.7]{};
          \draw(spec.in 2) node[ocirc, scale=0.7]{};
          \draw(spec.west) ++ (-0.6, 0) node[anchor=south]{Coax};

          % DFS
          \draw(shackswitch.west) -- ++ (-2, 0)
          to[amp, invert] ++ (-2, 0)
          to[amp, invert, name=dfslna1] ++ (-2, 0)
          -| ++ (-1, 0)
          coordinate(dfs)
          -- ++ (0, 0.5)
          node[bareantenna, anchor=south](dfsantenna){};
          % \draw(dfslna1.east) ++ (0.5, 0.5) node[anchor=south](){PMA3-14LN+};

          % EMS
          % \draw(spec.north) -- ++ (0, 1.5)
          % to[amp, invert, name=emslna2] ++ (0, 2)
          % to[amp, invert, name=emslna1] ++ (0, 2)
          % node[bareantenna, anchor=south](emsantenna){};
          \draw(shackswitch.north) -- ++ (0, 2)
          node[twoportshape, name=s1]{} -- ++ (0, 0.8)
          to[amp, boxed, invert] ++ (0, 1) -- ++ (0, 0.8)
          node[twoportshape, name=s2]{} -- ++ (0, 0.8)
          node[bareantenna, anchor=south](emsantenna){};
          % \draw(spec.north) ++ (0, 0.75) node[anchor=west]{Coax};
          \draw(s1.center) node[spdt, scale=0.6, rotate=90]{};
          \draw(s2.center) node[spdt, xscale=-1, scale=0.6, rotate=270]{};
          % \draw(emslna1) ++ (0, -0.7) node[anchor=west](emslnalabel){PMA3-14LN+};

          % Motor stuff
          \draw(computer.south) |- ++ (-8.5, -1)
          node[qfpchip, anchor=west, external pins width=0, hide numbers, name=acr, scale=0.8, align=center]{Motor\\Ctrl.};
          \draw(computer.south) ++ (0, -1)
          node[anchor=south east](seriallabel){RS-485};
          \draw([yshift=-8pt]acr.west) -- ++ (-2, 0)
          node[elmech, rotate=90, anchor=bottom](m2){};
          \draw(m2) [->, dashed] -| ([yshift=-10pt]dfs.center);
          \draw([yshift=8pt]acr.west) -- ++ (-0.6, 0)
          node[elmech, rotate=90, anchor=bottom](m1){};
          %\draw(m1) [->] -- ++ (3, 0) -- ([yshift=-10pt, xshift=-10pt]dfs.center);
          \path[name path = border3](dfs.center) -- ++ (0, -2);
          \path[name path = line3, overlay](m1.top) -- ++ (-5, 0);
          \draw[name intersections={of=border3 and line3}, dashed] (m1.top) -- (intersection-1)
          node[diamondpole]{};
          \draw(m1.center) node{Az};
          \draw(m2.center) node{El};

          % Boxes
          \node[draw, rectangle, dashed, fit=(dfsantenna) (acr), inner sep=10](dfsbox){};
          \node[draw, rectangle, dashed, fit=(emsantenna) (s1), inner sep=10](emsbox){};
          \node[draw, rectangle, dashed, fit=(shackswitch) (computer), inner sep=10](shackbox){};

          % Box Labels
          \draw(dfsbox.north) node[anchor=south]{RF - DFS};
          \draw(emsbox.west) node[anchor=east]{RF - EMS};
          \draw(shackbox.north east) node[anchor=south east]{RFI Shack};
      \end{circuitikz}
  \caption{System block diagram (not including PLC).}\label{fig:sysblock2}
  \end{center}
\end{figure}

\subsection{RF Signal Chain}
asdf

\begin{figure}[!ht]
    \begin{center}
        \begin{circuitikz}
            % \draw[help lines, dashed] grid (14, 3);
            % \draw[help lines, dashed] grid (14, -3);

            \draw(0, 0) node[bnc](port1){N9040b};
            \draw(port1.hot) -- ++ (1, 0)
            node[rotary switch = 2 in 30 wiper 20, anchor=in](sw1){};
            
            \draw(sw1.out 1) -- ++ (0.5, 0) 
            to[twoport, name=emscableloss, fill=green!20] ++ (0, 2.35)
            -- ++ (0.75, 0)
            node[rotary switch = 2 in 30 wiper 20, anchor=in](sw2){};
            \draw (emscableloss.center) node[resistorshape, anchor = center, rotate = 90, scale = 0.8]{};
            
            % EMS low band
            \draw(sw2.out 1) -| ++ (0.75, 0.5)
            -- ++ (0.5, 0)
            to[amp, invert, t=\ctikzflipx{\footnotesize LNA}] ++ (2.75, 0)
            coordinate(emsref1)
            to[amp, invert, t=\ctikzflipx{\footnotesize LNA}] ++ (2.75, 0)
            -- ++ (0.5, 0)
            coordinate(p1);
            
            % EMS downconverted stage
            \draw(sw2.out 2) -| ++ (0.75, -0.5) coordinate(asdf)
            -- ++ (0.25 ,0)
            to[amp, invert, name=ifamp, t=\ctikzflipx{IF}] ++ (1.5, 0)
            -- ++ (0.25, 0) node[mixer, anchor=w, name=mixer1]{};
            \draw(mixer1.e) -- ++ (0.25, 0)
            to[bandpass] ++ (1.5, 0)
            to[amp, invert, name=rfamp, t=\ctikzflipx{RF}] ++ (1.5, 0) -- ++ (0.25, 0)
            coordinate(p2);
            \draw(mixer1.s) -- ++ (0, -0.5)
            node[anchor=north](pll){LO};
            
            % DFS
            \draw(sw1.out 2) -| ++ (0.5, 0)
            to[twoport, name=dfscableloss, fill=green!20] (current subpath start -| asdf)
            -- (current subpath start -| p2)
            node[ampshape, xscale=-1, anchor=east, fill=yellow!30, t=\ctikzflipx{\footnotesize LNA}](dfsref1){};
            \draw(dfsref1.west)
            to[amp, invert, t=\ctikzflipx{\footnotesize LNA}] ++ (2, 0)
            node[bnc, xscale=-1](port3){\ctikzflipx{DFS}};
            \draw (dfscableloss.center) node[resistorshape, anchor = center, rotate = 90, scale = 0.8]{};
            
            % EMS port
            \draw($(p1)!0.5!(p2)$) ++ (1, 0)
            node[rotary switch = 2 in 30 wiper 20, xscale=-1](sw3){};
            \draw(sw3.out 1) -| (p1);
            \draw(sw3.out 2) -| (p2);
            \draw(sw3.in) -- (sw3.in -| port3)
            node[bnc, xscale=-1](port2){\ctikzflipx{EMS}};

            % Switch labels and arrows
            \draw(sw1.north) node[anchor=south](sw1lab){S1};
            \draw(sw2.north) node[anchor=south](sw2lab){S2};
            \draw(sw3.north) node[anchor=south](sw3lab){S3};
            \draw(sw1lab.north) [dashed, <-] -- ++ (0, 3.25) coordinate(sw1plclabel) node[anchor=south]{PLC};
            \draw(sw2lab.north) [dashed, <-] -- (sw2lab.north |- sw1plclabel) node[anchor=south]{PLC};
            \draw(sw3lab.north) [dashed, <-] -- (sw3lab.north |- sw1plclabel) node[anchor=south]{PLC};

            % Phase 3 box            
            \node[draw, rectangle, dashed, fit=(ifamp) (rfamp) (pll), inner sep=8](ph3box){};

            % Labels
            \draw([yshift=2pt]dfscableloss.north)[<-] -- ++ (0, 0.25) node[anchor=south](cablelosslabel){Cable loss};
            \draw(cablelosslabel.north)[->] to [bend right=45] ([xshift=2pt]emscableloss.south);
            \draw(ph3box.south east) node[anchor=south east]{Future Upgrade};
            % \draw(emsref1.north |- sw1plclabel) node[anchor=south]{PMA3-14LN+};
            % \draw(dfsref1.north) ++ (0.8, 0.2) node[anchor=south]{PMA3-14LN+};
        \end{circuitikz}
    \caption{Phase 2 Block Diagram}\label{fig:ph3ampblock}
    \end{center}
\end{figure}
\begin{figure}[!ht]
  \begin{center}
    %% Creator: Matplotlib, PGF backend
%%
%% To include the figure in your LaTeX document, write
%%   \input{<filename>.pgf}
%%
%% Make sure the required packages are loaded in your preamble
%%   \usepackage{pgf}
%%
%% Also ensure that all the required font packages are loaded; for instance,
%% the lmodern package is sometimes necessary when using math font.
%%   \usepackage{lmodern}
%%
%% Figures using additional raster images can only be included by \input if
%% they are in the same directory as the main LaTeX file. For loading figures
%% from other directories you can use the `import` package
%%   \usepackage{import}
%%
%% and then include the figures with
%%   \import{<path to file>}{<filename>.pgf}
%%
%% Matplotlib used the following preamble
%%   \def\mathdefault#1{#1}
%%   \everymath=\expandafter{\the\everymath\displaystyle}
%%   
%%   \ifdefined\pdftexversion\else  % non-pdftex case.
%%     \usepackage{fontspec}
%%   \fi
%%   \makeatletter\@ifpackageloaded{underscore}{}{\usepackage[strings]{underscore}}\makeatother
%%
\begingroup%
\makeatletter%
\begin{pgfpicture}%
\pgfpathrectangle{\pgfpointorigin}{\pgfqpoint{4.416023in}{3.339900in}}%
\pgfusepath{use as bounding box, clip}%
\begin{pgfscope}%
\pgfsetbuttcap%
\pgfsetmiterjoin%
\definecolor{currentfill}{rgb}{1.000000,1.000000,1.000000}%
\pgfsetfillcolor{currentfill}%
\pgfsetlinewidth{0.000000pt}%
\definecolor{currentstroke}{rgb}{1.000000,1.000000,1.000000}%
\pgfsetstrokecolor{currentstroke}%
\pgfsetdash{}{0pt}%
\pgfpathmoveto{\pgfqpoint{0.000000in}{0.000000in}}%
\pgfpathlineto{\pgfqpoint{4.416023in}{0.000000in}}%
\pgfpathlineto{\pgfqpoint{4.416023in}{3.339900in}}%
\pgfpathlineto{\pgfqpoint{0.000000in}{3.339900in}}%
\pgfpathlineto{\pgfqpoint{0.000000in}{0.000000in}}%
\pgfpathclose%
\pgfusepath{fill}%
\end{pgfscope}%
\begin{pgfscope}%
\pgfsetbuttcap%
\pgfsetmiterjoin%
\definecolor{currentfill}{rgb}{1.000000,1.000000,1.000000}%
\pgfsetfillcolor{currentfill}%
\pgfsetlinewidth{0.000000pt}%
\definecolor{currentstroke}{rgb}{0.000000,0.000000,0.000000}%
\pgfsetstrokecolor{currentstroke}%
\pgfsetstrokeopacity{0.000000}%
\pgfsetdash{}{0pt}%
\pgfpathmoveto{\pgfqpoint{0.643596in}{0.515000in}}%
\pgfpathlineto{\pgfqpoint{4.142411in}{0.515000in}}%
\pgfpathlineto{\pgfqpoint{4.142411in}{3.040900in}}%
\pgfpathlineto{\pgfqpoint{0.643596in}{3.040900in}}%
\pgfpathlineto{\pgfqpoint{0.643596in}{0.515000in}}%
\pgfpathclose%
\pgfusepath{fill}%
\end{pgfscope}%
\begin{pgfscope}%
\pgfpathrectangle{\pgfqpoint{0.643596in}{0.515000in}}{\pgfqpoint{3.498815in}{2.525900in}}%
\pgfusepath{clip}%
\pgfsetrectcap%
\pgfsetroundjoin%
\pgfsetlinewidth{0.803000pt}%
\definecolor{currentstroke}{rgb}{0.690196,0.690196,0.690196}%
\pgfsetstrokecolor{currentstroke}%
\pgfsetdash{}{0pt}%
\pgfpathmoveto{\pgfqpoint{0.643596in}{0.515000in}}%
\pgfpathlineto{\pgfqpoint{0.643596in}{3.040900in}}%
\pgfusepath{stroke}%
\end{pgfscope}%
\begin{pgfscope}%
\pgfsetbuttcap%
\pgfsetroundjoin%
\definecolor{currentfill}{rgb}{0.000000,0.000000,0.000000}%
\pgfsetfillcolor{currentfill}%
\pgfsetlinewidth{0.803000pt}%
\definecolor{currentstroke}{rgb}{0.000000,0.000000,0.000000}%
\pgfsetstrokecolor{currentstroke}%
\pgfsetdash{}{0pt}%
\pgfsys@defobject{currentmarker}{\pgfqpoint{0.000000in}{-0.048611in}}{\pgfqpoint{0.000000in}{0.000000in}}{%
\pgfpathmoveto{\pgfqpoint{0.000000in}{0.000000in}}%
\pgfpathlineto{\pgfqpoint{0.000000in}{-0.048611in}}%
\pgfusepath{stroke,fill}%
}%
\begin{pgfscope}%
\pgfsys@transformshift{0.643596in}{0.515000in}%
\pgfsys@useobject{currentmarker}{}%
\end{pgfscope}%
\end{pgfscope}%
\begin{pgfscope}%
\definecolor{textcolor}{rgb}{0.000000,0.000000,0.000000}%
\pgfsetstrokecolor{textcolor}%
\pgfsetfillcolor{textcolor}%
\pgftext[x=0.643596in,y=0.417777in,,top]{\color{textcolor}{\rmfamily\fontsize{10.000000}{12.000000}\selectfont\catcode`\^=\active\def^{\ifmmode\sp\else\^{}\fi}\catcode`\%=\active\def%{\%}$\mathdefault{0}$}}%
\end{pgfscope}%
\begin{pgfscope}%
\pgfpathrectangle{\pgfqpoint{0.643596in}{0.515000in}}{\pgfqpoint{3.498815in}{2.525900in}}%
\pgfusepath{clip}%
\pgfsetrectcap%
\pgfsetroundjoin%
\pgfsetlinewidth{0.803000pt}%
\definecolor{currentstroke}{rgb}{0.690196,0.690196,0.690196}%
\pgfsetstrokecolor{currentstroke}%
\pgfsetdash{}{0pt}%
\pgfpathmoveto{\pgfqpoint{1.343359in}{0.515000in}}%
\pgfpathlineto{\pgfqpoint{1.343359in}{3.040900in}}%
\pgfusepath{stroke}%
\end{pgfscope}%
\begin{pgfscope}%
\pgfsetbuttcap%
\pgfsetroundjoin%
\definecolor{currentfill}{rgb}{0.000000,0.000000,0.000000}%
\pgfsetfillcolor{currentfill}%
\pgfsetlinewidth{0.803000pt}%
\definecolor{currentstroke}{rgb}{0.000000,0.000000,0.000000}%
\pgfsetstrokecolor{currentstroke}%
\pgfsetdash{}{0pt}%
\pgfsys@defobject{currentmarker}{\pgfqpoint{0.000000in}{-0.048611in}}{\pgfqpoint{0.000000in}{0.000000in}}{%
\pgfpathmoveto{\pgfqpoint{0.000000in}{0.000000in}}%
\pgfpathlineto{\pgfqpoint{0.000000in}{-0.048611in}}%
\pgfusepath{stroke,fill}%
}%
\begin{pgfscope}%
\pgfsys@transformshift{1.343359in}{0.515000in}%
\pgfsys@useobject{currentmarker}{}%
\end{pgfscope}%
\end{pgfscope}%
\begin{pgfscope}%
\definecolor{textcolor}{rgb}{0.000000,0.000000,0.000000}%
\pgfsetstrokecolor{textcolor}%
\pgfsetfillcolor{textcolor}%
\pgftext[x=1.343359in,y=0.417777in,,top]{\color{textcolor}{\rmfamily\fontsize{10.000000}{12.000000}\selectfont\catcode`\^=\active\def^{\ifmmode\sp\else\^{}\fi}\catcode`\%=\active\def%{\%}$\mathdefault{2000}$}}%
\end{pgfscope}%
\begin{pgfscope}%
\pgfpathrectangle{\pgfqpoint{0.643596in}{0.515000in}}{\pgfqpoint{3.498815in}{2.525900in}}%
\pgfusepath{clip}%
\pgfsetrectcap%
\pgfsetroundjoin%
\pgfsetlinewidth{0.803000pt}%
\definecolor{currentstroke}{rgb}{0.690196,0.690196,0.690196}%
\pgfsetstrokecolor{currentstroke}%
\pgfsetdash{}{0pt}%
\pgfpathmoveto{\pgfqpoint{2.043122in}{0.515000in}}%
\pgfpathlineto{\pgfqpoint{2.043122in}{3.040900in}}%
\pgfusepath{stroke}%
\end{pgfscope}%
\begin{pgfscope}%
\pgfsetbuttcap%
\pgfsetroundjoin%
\definecolor{currentfill}{rgb}{0.000000,0.000000,0.000000}%
\pgfsetfillcolor{currentfill}%
\pgfsetlinewidth{0.803000pt}%
\definecolor{currentstroke}{rgb}{0.000000,0.000000,0.000000}%
\pgfsetstrokecolor{currentstroke}%
\pgfsetdash{}{0pt}%
\pgfsys@defobject{currentmarker}{\pgfqpoint{0.000000in}{-0.048611in}}{\pgfqpoint{0.000000in}{0.000000in}}{%
\pgfpathmoveto{\pgfqpoint{0.000000in}{0.000000in}}%
\pgfpathlineto{\pgfqpoint{0.000000in}{-0.048611in}}%
\pgfusepath{stroke,fill}%
}%
\begin{pgfscope}%
\pgfsys@transformshift{2.043122in}{0.515000in}%
\pgfsys@useobject{currentmarker}{}%
\end{pgfscope}%
\end{pgfscope}%
\begin{pgfscope}%
\definecolor{textcolor}{rgb}{0.000000,0.000000,0.000000}%
\pgfsetstrokecolor{textcolor}%
\pgfsetfillcolor{textcolor}%
\pgftext[x=2.043122in,y=0.417777in,,top]{\color{textcolor}{\rmfamily\fontsize{10.000000}{12.000000}\selectfont\catcode`\^=\active\def^{\ifmmode\sp\else\^{}\fi}\catcode`\%=\active\def%{\%}$\mathdefault{4000}$}}%
\end{pgfscope}%
\begin{pgfscope}%
\pgfpathrectangle{\pgfqpoint{0.643596in}{0.515000in}}{\pgfqpoint{3.498815in}{2.525900in}}%
\pgfusepath{clip}%
\pgfsetrectcap%
\pgfsetroundjoin%
\pgfsetlinewidth{0.803000pt}%
\definecolor{currentstroke}{rgb}{0.690196,0.690196,0.690196}%
\pgfsetstrokecolor{currentstroke}%
\pgfsetdash{}{0pt}%
\pgfpathmoveto{\pgfqpoint{2.742885in}{0.515000in}}%
\pgfpathlineto{\pgfqpoint{2.742885in}{3.040900in}}%
\pgfusepath{stroke}%
\end{pgfscope}%
\begin{pgfscope}%
\pgfsetbuttcap%
\pgfsetroundjoin%
\definecolor{currentfill}{rgb}{0.000000,0.000000,0.000000}%
\pgfsetfillcolor{currentfill}%
\pgfsetlinewidth{0.803000pt}%
\definecolor{currentstroke}{rgb}{0.000000,0.000000,0.000000}%
\pgfsetstrokecolor{currentstroke}%
\pgfsetdash{}{0pt}%
\pgfsys@defobject{currentmarker}{\pgfqpoint{0.000000in}{-0.048611in}}{\pgfqpoint{0.000000in}{0.000000in}}{%
\pgfpathmoveto{\pgfqpoint{0.000000in}{0.000000in}}%
\pgfpathlineto{\pgfqpoint{0.000000in}{-0.048611in}}%
\pgfusepath{stroke,fill}%
}%
\begin{pgfscope}%
\pgfsys@transformshift{2.742885in}{0.515000in}%
\pgfsys@useobject{currentmarker}{}%
\end{pgfscope}%
\end{pgfscope}%
\begin{pgfscope}%
\definecolor{textcolor}{rgb}{0.000000,0.000000,0.000000}%
\pgfsetstrokecolor{textcolor}%
\pgfsetfillcolor{textcolor}%
\pgftext[x=2.742885in,y=0.417777in,,top]{\color{textcolor}{\rmfamily\fontsize{10.000000}{12.000000}\selectfont\catcode`\^=\active\def^{\ifmmode\sp\else\^{}\fi}\catcode`\%=\active\def%{\%}$\mathdefault{6000}$}}%
\end{pgfscope}%
\begin{pgfscope}%
\pgfpathrectangle{\pgfqpoint{0.643596in}{0.515000in}}{\pgfqpoint{3.498815in}{2.525900in}}%
\pgfusepath{clip}%
\pgfsetrectcap%
\pgfsetroundjoin%
\pgfsetlinewidth{0.803000pt}%
\definecolor{currentstroke}{rgb}{0.690196,0.690196,0.690196}%
\pgfsetstrokecolor{currentstroke}%
\pgfsetdash{}{0pt}%
\pgfpathmoveto{\pgfqpoint{3.442648in}{0.515000in}}%
\pgfpathlineto{\pgfqpoint{3.442648in}{3.040900in}}%
\pgfusepath{stroke}%
\end{pgfscope}%
\begin{pgfscope}%
\pgfsetbuttcap%
\pgfsetroundjoin%
\definecolor{currentfill}{rgb}{0.000000,0.000000,0.000000}%
\pgfsetfillcolor{currentfill}%
\pgfsetlinewidth{0.803000pt}%
\definecolor{currentstroke}{rgb}{0.000000,0.000000,0.000000}%
\pgfsetstrokecolor{currentstroke}%
\pgfsetdash{}{0pt}%
\pgfsys@defobject{currentmarker}{\pgfqpoint{0.000000in}{-0.048611in}}{\pgfqpoint{0.000000in}{0.000000in}}{%
\pgfpathmoveto{\pgfqpoint{0.000000in}{0.000000in}}%
\pgfpathlineto{\pgfqpoint{0.000000in}{-0.048611in}}%
\pgfusepath{stroke,fill}%
}%
\begin{pgfscope}%
\pgfsys@transformshift{3.442648in}{0.515000in}%
\pgfsys@useobject{currentmarker}{}%
\end{pgfscope}%
\end{pgfscope}%
\begin{pgfscope}%
\definecolor{textcolor}{rgb}{0.000000,0.000000,0.000000}%
\pgfsetstrokecolor{textcolor}%
\pgfsetfillcolor{textcolor}%
\pgftext[x=3.442648in,y=0.417777in,,top]{\color{textcolor}{\rmfamily\fontsize{10.000000}{12.000000}\selectfont\catcode`\^=\active\def^{\ifmmode\sp\else\^{}\fi}\catcode`\%=\active\def%{\%}$\mathdefault{8000}$}}%
\end{pgfscope}%
\begin{pgfscope}%
\pgfpathrectangle{\pgfqpoint{0.643596in}{0.515000in}}{\pgfqpoint{3.498815in}{2.525900in}}%
\pgfusepath{clip}%
\pgfsetrectcap%
\pgfsetroundjoin%
\pgfsetlinewidth{0.803000pt}%
\definecolor{currentstroke}{rgb}{0.690196,0.690196,0.690196}%
\pgfsetstrokecolor{currentstroke}%
\pgfsetdash{}{0pt}%
\pgfpathmoveto{\pgfqpoint{4.142411in}{0.515000in}}%
\pgfpathlineto{\pgfqpoint{4.142411in}{3.040900in}}%
\pgfusepath{stroke}%
\end{pgfscope}%
\begin{pgfscope}%
\pgfsetbuttcap%
\pgfsetroundjoin%
\definecolor{currentfill}{rgb}{0.000000,0.000000,0.000000}%
\pgfsetfillcolor{currentfill}%
\pgfsetlinewidth{0.803000pt}%
\definecolor{currentstroke}{rgb}{0.000000,0.000000,0.000000}%
\pgfsetstrokecolor{currentstroke}%
\pgfsetdash{}{0pt}%
\pgfsys@defobject{currentmarker}{\pgfqpoint{0.000000in}{-0.048611in}}{\pgfqpoint{0.000000in}{0.000000in}}{%
\pgfpathmoveto{\pgfqpoint{0.000000in}{0.000000in}}%
\pgfpathlineto{\pgfqpoint{0.000000in}{-0.048611in}}%
\pgfusepath{stroke,fill}%
}%
\begin{pgfscope}%
\pgfsys@transformshift{4.142411in}{0.515000in}%
\pgfsys@useobject{currentmarker}{}%
\end{pgfscope}%
\end{pgfscope}%
\begin{pgfscope}%
\definecolor{textcolor}{rgb}{0.000000,0.000000,0.000000}%
\pgfsetstrokecolor{textcolor}%
\pgfsetfillcolor{textcolor}%
\pgftext[x=4.142411in,y=0.417777in,,top]{\color{textcolor}{\rmfamily\fontsize{10.000000}{12.000000}\selectfont\catcode`\^=\active\def^{\ifmmode\sp\else\^{}\fi}\catcode`\%=\active\def%{\%}$\mathdefault{10000}$}}%
\end{pgfscope}%
\begin{pgfscope}%
\pgfsetbuttcap%
\pgfsetroundjoin%
\definecolor{currentfill}{rgb}{0.000000,0.000000,0.000000}%
\pgfsetfillcolor{currentfill}%
\pgfsetlinewidth{0.602250pt}%
\definecolor{currentstroke}{rgb}{0.000000,0.000000,0.000000}%
\pgfsetstrokecolor{currentstroke}%
\pgfsetdash{}{0pt}%
\pgfsys@defobject{currentmarker}{\pgfqpoint{0.000000in}{-0.027778in}}{\pgfqpoint{0.000000in}{0.000000in}}{%
\pgfpathmoveto{\pgfqpoint{0.000000in}{0.000000in}}%
\pgfpathlineto{\pgfqpoint{0.000000in}{-0.027778in}}%
\pgfusepath{stroke,fill}%
}%
\begin{pgfscope}%
\pgfsys@transformshift{0.818537in}{0.515000in}%
\pgfsys@useobject{currentmarker}{}%
\end{pgfscope}%
\end{pgfscope}%
\begin{pgfscope}%
\pgfsetbuttcap%
\pgfsetroundjoin%
\definecolor{currentfill}{rgb}{0.000000,0.000000,0.000000}%
\pgfsetfillcolor{currentfill}%
\pgfsetlinewidth{0.602250pt}%
\definecolor{currentstroke}{rgb}{0.000000,0.000000,0.000000}%
\pgfsetstrokecolor{currentstroke}%
\pgfsetdash{}{0pt}%
\pgfsys@defobject{currentmarker}{\pgfqpoint{0.000000in}{-0.027778in}}{\pgfqpoint{0.000000in}{0.000000in}}{%
\pgfpathmoveto{\pgfqpoint{0.000000in}{0.000000in}}%
\pgfpathlineto{\pgfqpoint{0.000000in}{-0.027778in}}%
\pgfusepath{stroke,fill}%
}%
\begin{pgfscope}%
\pgfsys@transformshift{0.993477in}{0.515000in}%
\pgfsys@useobject{currentmarker}{}%
\end{pgfscope}%
\end{pgfscope}%
\begin{pgfscope}%
\pgfsetbuttcap%
\pgfsetroundjoin%
\definecolor{currentfill}{rgb}{0.000000,0.000000,0.000000}%
\pgfsetfillcolor{currentfill}%
\pgfsetlinewidth{0.602250pt}%
\definecolor{currentstroke}{rgb}{0.000000,0.000000,0.000000}%
\pgfsetstrokecolor{currentstroke}%
\pgfsetdash{}{0pt}%
\pgfsys@defobject{currentmarker}{\pgfqpoint{0.000000in}{-0.027778in}}{\pgfqpoint{0.000000in}{0.000000in}}{%
\pgfpathmoveto{\pgfqpoint{0.000000in}{0.000000in}}%
\pgfpathlineto{\pgfqpoint{0.000000in}{-0.027778in}}%
\pgfusepath{stroke,fill}%
}%
\begin{pgfscope}%
\pgfsys@transformshift{1.168418in}{0.515000in}%
\pgfsys@useobject{currentmarker}{}%
\end{pgfscope}%
\end{pgfscope}%
\begin{pgfscope}%
\pgfsetbuttcap%
\pgfsetroundjoin%
\definecolor{currentfill}{rgb}{0.000000,0.000000,0.000000}%
\pgfsetfillcolor{currentfill}%
\pgfsetlinewidth{0.602250pt}%
\definecolor{currentstroke}{rgb}{0.000000,0.000000,0.000000}%
\pgfsetstrokecolor{currentstroke}%
\pgfsetdash{}{0pt}%
\pgfsys@defobject{currentmarker}{\pgfqpoint{0.000000in}{-0.027778in}}{\pgfqpoint{0.000000in}{0.000000in}}{%
\pgfpathmoveto{\pgfqpoint{0.000000in}{0.000000in}}%
\pgfpathlineto{\pgfqpoint{0.000000in}{-0.027778in}}%
\pgfusepath{stroke,fill}%
}%
\begin{pgfscope}%
\pgfsys@transformshift{1.518300in}{0.515000in}%
\pgfsys@useobject{currentmarker}{}%
\end{pgfscope}%
\end{pgfscope}%
\begin{pgfscope}%
\pgfsetbuttcap%
\pgfsetroundjoin%
\definecolor{currentfill}{rgb}{0.000000,0.000000,0.000000}%
\pgfsetfillcolor{currentfill}%
\pgfsetlinewidth{0.602250pt}%
\definecolor{currentstroke}{rgb}{0.000000,0.000000,0.000000}%
\pgfsetstrokecolor{currentstroke}%
\pgfsetdash{}{0pt}%
\pgfsys@defobject{currentmarker}{\pgfqpoint{0.000000in}{-0.027778in}}{\pgfqpoint{0.000000in}{0.000000in}}{%
\pgfpathmoveto{\pgfqpoint{0.000000in}{0.000000in}}%
\pgfpathlineto{\pgfqpoint{0.000000in}{-0.027778in}}%
\pgfusepath{stroke,fill}%
}%
\begin{pgfscope}%
\pgfsys@transformshift{1.693240in}{0.515000in}%
\pgfsys@useobject{currentmarker}{}%
\end{pgfscope}%
\end{pgfscope}%
\begin{pgfscope}%
\pgfsetbuttcap%
\pgfsetroundjoin%
\definecolor{currentfill}{rgb}{0.000000,0.000000,0.000000}%
\pgfsetfillcolor{currentfill}%
\pgfsetlinewidth{0.602250pt}%
\definecolor{currentstroke}{rgb}{0.000000,0.000000,0.000000}%
\pgfsetstrokecolor{currentstroke}%
\pgfsetdash{}{0pt}%
\pgfsys@defobject{currentmarker}{\pgfqpoint{0.000000in}{-0.027778in}}{\pgfqpoint{0.000000in}{0.000000in}}{%
\pgfpathmoveto{\pgfqpoint{0.000000in}{0.000000in}}%
\pgfpathlineto{\pgfqpoint{0.000000in}{-0.027778in}}%
\pgfusepath{stroke,fill}%
}%
\begin{pgfscope}%
\pgfsys@transformshift{1.868181in}{0.515000in}%
\pgfsys@useobject{currentmarker}{}%
\end{pgfscope}%
\end{pgfscope}%
\begin{pgfscope}%
\pgfsetbuttcap%
\pgfsetroundjoin%
\definecolor{currentfill}{rgb}{0.000000,0.000000,0.000000}%
\pgfsetfillcolor{currentfill}%
\pgfsetlinewidth{0.602250pt}%
\definecolor{currentstroke}{rgb}{0.000000,0.000000,0.000000}%
\pgfsetstrokecolor{currentstroke}%
\pgfsetdash{}{0pt}%
\pgfsys@defobject{currentmarker}{\pgfqpoint{0.000000in}{-0.027778in}}{\pgfqpoint{0.000000in}{0.000000in}}{%
\pgfpathmoveto{\pgfqpoint{0.000000in}{0.000000in}}%
\pgfpathlineto{\pgfqpoint{0.000000in}{-0.027778in}}%
\pgfusepath{stroke,fill}%
}%
\begin{pgfscope}%
\pgfsys@transformshift{2.218063in}{0.515000in}%
\pgfsys@useobject{currentmarker}{}%
\end{pgfscope}%
\end{pgfscope}%
\begin{pgfscope}%
\pgfsetbuttcap%
\pgfsetroundjoin%
\definecolor{currentfill}{rgb}{0.000000,0.000000,0.000000}%
\pgfsetfillcolor{currentfill}%
\pgfsetlinewidth{0.602250pt}%
\definecolor{currentstroke}{rgb}{0.000000,0.000000,0.000000}%
\pgfsetstrokecolor{currentstroke}%
\pgfsetdash{}{0pt}%
\pgfsys@defobject{currentmarker}{\pgfqpoint{0.000000in}{-0.027778in}}{\pgfqpoint{0.000000in}{0.000000in}}{%
\pgfpathmoveto{\pgfqpoint{0.000000in}{0.000000in}}%
\pgfpathlineto{\pgfqpoint{0.000000in}{-0.027778in}}%
\pgfusepath{stroke,fill}%
}%
\begin{pgfscope}%
\pgfsys@transformshift{2.393003in}{0.515000in}%
\pgfsys@useobject{currentmarker}{}%
\end{pgfscope}%
\end{pgfscope}%
\begin{pgfscope}%
\pgfsetbuttcap%
\pgfsetroundjoin%
\definecolor{currentfill}{rgb}{0.000000,0.000000,0.000000}%
\pgfsetfillcolor{currentfill}%
\pgfsetlinewidth{0.602250pt}%
\definecolor{currentstroke}{rgb}{0.000000,0.000000,0.000000}%
\pgfsetstrokecolor{currentstroke}%
\pgfsetdash{}{0pt}%
\pgfsys@defobject{currentmarker}{\pgfqpoint{0.000000in}{-0.027778in}}{\pgfqpoint{0.000000in}{0.000000in}}{%
\pgfpathmoveto{\pgfqpoint{0.000000in}{0.000000in}}%
\pgfpathlineto{\pgfqpoint{0.000000in}{-0.027778in}}%
\pgfusepath{stroke,fill}%
}%
\begin{pgfscope}%
\pgfsys@transformshift{2.567944in}{0.515000in}%
\pgfsys@useobject{currentmarker}{}%
\end{pgfscope}%
\end{pgfscope}%
\begin{pgfscope}%
\pgfsetbuttcap%
\pgfsetroundjoin%
\definecolor{currentfill}{rgb}{0.000000,0.000000,0.000000}%
\pgfsetfillcolor{currentfill}%
\pgfsetlinewidth{0.602250pt}%
\definecolor{currentstroke}{rgb}{0.000000,0.000000,0.000000}%
\pgfsetstrokecolor{currentstroke}%
\pgfsetdash{}{0pt}%
\pgfsys@defobject{currentmarker}{\pgfqpoint{0.000000in}{-0.027778in}}{\pgfqpoint{0.000000in}{0.000000in}}{%
\pgfpathmoveto{\pgfqpoint{0.000000in}{0.000000in}}%
\pgfpathlineto{\pgfqpoint{0.000000in}{-0.027778in}}%
\pgfusepath{stroke,fill}%
}%
\begin{pgfscope}%
\pgfsys@transformshift{2.917826in}{0.515000in}%
\pgfsys@useobject{currentmarker}{}%
\end{pgfscope}%
\end{pgfscope}%
\begin{pgfscope}%
\pgfsetbuttcap%
\pgfsetroundjoin%
\definecolor{currentfill}{rgb}{0.000000,0.000000,0.000000}%
\pgfsetfillcolor{currentfill}%
\pgfsetlinewidth{0.602250pt}%
\definecolor{currentstroke}{rgb}{0.000000,0.000000,0.000000}%
\pgfsetstrokecolor{currentstroke}%
\pgfsetdash{}{0pt}%
\pgfsys@defobject{currentmarker}{\pgfqpoint{0.000000in}{-0.027778in}}{\pgfqpoint{0.000000in}{0.000000in}}{%
\pgfpathmoveto{\pgfqpoint{0.000000in}{0.000000in}}%
\pgfpathlineto{\pgfqpoint{0.000000in}{-0.027778in}}%
\pgfusepath{stroke,fill}%
}%
\begin{pgfscope}%
\pgfsys@transformshift{3.092766in}{0.515000in}%
\pgfsys@useobject{currentmarker}{}%
\end{pgfscope}%
\end{pgfscope}%
\begin{pgfscope}%
\pgfsetbuttcap%
\pgfsetroundjoin%
\definecolor{currentfill}{rgb}{0.000000,0.000000,0.000000}%
\pgfsetfillcolor{currentfill}%
\pgfsetlinewidth{0.602250pt}%
\definecolor{currentstroke}{rgb}{0.000000,0.000000,0.000000}%
\pgfsetstrokecolor{currentstroke}%
\pgfsetdash{}{0pt}%
\pgfsys@defobject{currentmarker}{\pgfqpoint{0.000000in}{-0.027778in}}{\pgfqpoint{0.000000in}{0.000000in}}{%
\pgfpathmoveto{\pgfqpoint{0.000000in}{0.000000in}}%
\pgfpathlineto{\pgfqpoint{0.000000in}{-0.027778in}}%
\pgfusepath{stroke,fill}%
}%
\begin{pgfscope}%
\pgfsys@transformshift{3.267707in}{0.515000in}%
\pgfsys@useobject{currentmarker}{}%
\end{pgfscope}%
\end{pgfscope}%
\begin{pgfscope}%
\pgfsetbuttcap%
\pgfsetroundjoin%
\definecolor{currentfill}{rgb}{0.000000,0.000000,0.000000}%
\pgfsetfillcolor{currentfill}%
\pgfsetlinewidth{0.602250pt}%
\definecolor{currentstroke}{rgb}{0.000000,0.000000,0.000000}%
\pgfsetstrokecolor{currentstroke}%
\pgfsetdash{}{0pt}%
\pgfsys@defobject{currentmarker}{\pgfqpoint{0.000000in}{-0.027778in}}{\pgfqpoint{0.000000in}{0.000000in}}{%
\pgfpathmoveto{\pgfqpoint{0.000000in}{0.000000in}}%
\pgfpathlineto{\pgfqpoint{0.000000in}{-0.027778in}}%
\pgfusepath{stroke,fill}%
}%
\begin{pgfscope}%
\pgfsys@transformshift{3.617589in}{0.515000in}%
\pgfsys@useobject{currentmarker}{}%
\end{pgfscope}%
\end{pgfscope}%
\begin{pgfscope}%
\pgfsetbuttcap%
\pgfsetroundjoin%
\definecolor{currentfill}{rgb}{0.000000,0.000000,0.000000}%
\pgfsetfillcolor{currentfill}%
\pgfsetlinewidth{0.602250pt}%
\definecolor{currentstroke}{rgb}{0.000000,0.000000,0.000000}%
\pgfsetstrokecolor{currentstroke}%
\pgfsetdash{}{0pt}%
\pgfsys@defobject{currentmarker}{\pgfqpoint{0.000000in}{-0.027778in}}{\pgfqpoint{0.000000in}{0.000000in}}{%
\pgfpathmoveto{\pgfqpoint{0.000000in}{0.000000in}}%
\pgfpathlineto{\pgfqpoint{0.000000in}{-0.027778in}}%
\pgfusepath{stroke,fill}%
}%
\begin{pgfscope}%
\pgfsys@transformshift{3.792530in}{0.515000in}%
\pgfsys@useobject{currentmarker}{}%
\end{pgfscope}%
\end{pgfscope}%
\begin{pgfscope}%
\pgfsetbuttcap%
\pgfsetroundjoin%
\definecolor{currentfill}{rgb}{0.000000,0.000000,0.000000}%
\pgfsetfillcolor{currentfill}%
\pgfsetlinewidth{0.602250pt}%
\definecolor{currentstroke}{rgb}{0.000000,0.000000,0.000000}%
\pgfsetstrokecolor{currentstroke}%
\pgfsetdash{}{0pt}%
\pgfsys@defobject{currentmarker}{\pgfqpoint{0.000000in}{-0.027778in}}{\pgfqpoint{0.000000in}{0.000000in}}{%
\pgfpathmoveto{\pgfqpoint{0.000000in}{0.000000in}}%
\pgfpathlineto{\pgfqpoint{0.000000in}{-0.027778in}}%
\pgfusepath{stroke,fill}%
}%
\begin{pgfscope}%
\pgfsys@transformshift{3.967470in}{0.515000in}%
\pgfsys@useobject{currentmarker}{}%
\end{pgfscope}%
\end{pgfscope}%
\begin{pgfscope}%
\definecolor{textcolor}{rgb}{0.000000,0.000000,0.000000}%
\pgfsetstrokecolor{textcolor}%
\pgfsetfillcolor{textcolor}%
\pgftext[x=2.393003in,y=0.238889in,,top]{\color{textcolor}{\rmfamily\fontsize{10.000000}{12.000000}\selectfont\catcode`\^=\active\def^{\ifmmode\sp\else\^{}\fi}\catcode`\%=\active\def%{\%}Frequency (MHz)}}%
\end{pgfscope}%
\begin{pgfscope}%
\pgfpathrectangle{\pgfqpoint{0.643596in}{0.515000in}}{\pgfqpoint{3.498815in}{2.525900in}}%
\pgfusepath{clip}%
\pgfsetrectcap%
\pgfsetroundjoin%
\pgfsetlinewidth{0.803000pt}%
\definecolor{currentstroke}{rgb}{0.690196,0.690196,0.690196}%
\pgfsetstrokecolor{currentstroke}%
\pgfsetdash{}{0pt}%
\pgfpathmoveto{\pgfqpoint{0.643596in}{0.515000in}}%
\pgfpathlineto{\pgfqpoint{4.142411in}{0.515000in}}%
\pgfusepath{stroke}%
\end{pgfscope}%
\begin{pgfscope}%
\pgfsetbuttcap%
\pgfsetroundjoin%
\definecolor{currentfill}{rgb}{0.000000,0.000000,0.000000}%
\pgfsetfillcolor{currentfill}%
\pgfsetlinewidth{0.803000pt}%
\definecolor{currentstroke}{rgb}{0.000000,0.000000,0.000000}%
\pgfsetstrokecolor{currentstroke}%
\pgfsetdash{}{0pt}%
\pgfsys@defobject{currentmarker}{\pgfqpoint{-0.048611in}{0.000000in}}{\pgfqpoint{-0.000000in}{0.000000in}}{%
\pgfpathmoveto{\pgfqpoint{-0.000000in}{0.000000in}}%
\pgfpathlineto{\pgfqpoint{-0.048611in}{0.000000in}}%
\pgfusepath{stroke,fill}%
}%
\begin{pgfscope}%
\pgfsys@transformshift{0.643596in}{0.515000in}%
\pgfsys@useobject{currentmarker}{}%
\end{pgfscope}%
\end{pgfscope}%
\begin{pgfscope}%
\definecolor{textcolor}{rgb}{0.000000,0.000000,0.000000}%
\pgfsetstrokecolor{textcolor}%
\pgfsetfillcolor{textcolor}%
\pgftext[x=0.299459in, y=0.466805in, left, base]{\color{textcolor}{\rmfamily\fontsize{10.000000}{12.000000}\selectfont\catcode`\^=\active\def^{\ifmmode\sp\else\^{}\fi}\catcode`\%=\active\def%{\%}$\mathdefault{\ensuremath{-}30}$}}%
\end{pgfscope}%
\begin{pgfscope}%
\pgfpathrectangle{\pgfqpoint{0.643596in}{0.515000in}}{\pgfqpoint{3.498815in}{2.525900in}}%
\pgfusepath{clip}%
\pgfsetrectcap%
\pgfsetroundjoin%
\pgfsetlinewidth{0.803000pt}%
\definecolor{currentstroke}{rgb}{0.690196,0.690196,0.690196}%
\pgfsetstrokecolor{currentstroke}%
\pgfsetdash{}{0pt}%
\pgfpathmoveto{\pgfqpoint{0.643596in}{0.935983in}}%
\pgfpathlineto{\pgfqpoint{4.142411in}{0.935983in}}%
\pgfusepath{stroke}%
\end{pgfscope}%
\begin{pgfscope}%
\pgfsetbuttcap%
\pgfsetroundjoin%
\definecolor{currentfill}{rgb}{0.000000,0.000000,0.000000}%
\pgfsetfillcolor{currentfill}%
\pgfsetlinewidth{0.803000pt}%
\definecolor{currentstroke}{rgb}{0.000000,0.000000,0.000000}%
\pgfsetstrokecolor{currentstroke}%
\pgfsetdash{}{0pt}%
\pgfsys@defobject{currentmarker}{\pgfqpoint{-0.048611in}{0.000000in}}{\pgfqpoint{-0.000000in}{0.000000in}}{%
\pgfpathmoveto{\pgfqpoint{-0.000000in}{0.000000in}}%
\pgfpathlineto{\pgfqpoint{-0.048611in}{0.000000in}}%
\pgfusepath{stroke,fill}%
}%
\begin{pgfscope}%
\pgfsys@transformshift{0.643596in}{0.935983in}%
\pgfsys@useobject{currentmarker}{}%
\end{pgfscope}%
\end{pgfscope}%
\begin{pgfscope}%
\definecolor{textcolor}{rgb}{0.000000,0.000000,0.000000}%
\pgfsetstrokecolor{textcolor}%
\pgfsetfillcolor{textcolor}%
\pgftext[x=0.299459in, y=0.887789in, left, base]{\color{textcolor}{\rmfamily\fontsize{10.000000}{12.000000}\selectfont\catcode`\^=\active\def^{\ifmmode\sp\else\^{}\fi}\catcode`\%=\active\def%{\%}$\mathdefault{\ensuremath{-}20}$}}%
\end{pgfscope}%
\begin{pgfscope}%
\pgfpathrectangle{\pgfqpoint{0.643596in}{0.515000in}}{\pgfqpoint{3.498815in}{2.525900in}}%
\pgfusepath{clip}%
\pgfsetrectcap%
\pgfsetroundjoin%
\pgfsetlinewidth{0.803000pt}%
\definecolor{currentstroke}{rgb}{0.690196,0.690196,0.690196}%
\pgfsetstrokecolor{currentstroke}%
\pgfsetdash{}{0pt}%
\pgfpathmoveto{\pgfqpoint{0.643596in}{1.356966in}}%
\pgfpathlineto{\pgfqpoint{4.142411in}{1.356966in}}%
\pgfusepath{stroke}%
\end{pgfscope}%
\begin{pgfscope}%
\pgfsetbuttcap%
\pgfsetroundjoin%
\definecolor{currentfill}{rgb}{0.000000,0.000000,0.000000}%
\pgfsetfillcolor{currentfill}%
\pgfsetlinewidth{0.803000pt}%
\definecolor{currentstroke}{rgb}{0.000000,0.000000,0.000000}%
\pgfsetstrokecolor{currentstroke}%
\pgfsetdash{}{0pt}%
\pgfsys@defobject{currentmarker}{\pgfqpoint{-0.048611in}{0.000000in}}{\pgfqpoint{-0.000000in}{0.000000in}}{%
\pgfpathmoveto{\pgfqpoint{-0.000000in}{0.000000in}}%
\pgfpathlineto{\pgfqpoint{-0.048611in}{0.000000in}}%
\pgfusepath{stroke,fill}%
}%
\begin{pgfscope}%
\pgfsys@transformshift{0.643596in}{1.356966in}%
\pgfsys@useobject{currentmarker}{}%
\end{pgfscope}%
\end{pgfscope}%
\begin{pgfscope}%
\definecolor{textcolor}{rgb}{0.000000,0.000000,0.000000}%
\pgfsetstrokecolor{textcolor}%
\pgfsetfillcolor{textcolor}%
\pgftext[x=0.299459in, y=1.308772in, left, base]{\color{textcolor}{\rmfamily\fontsize{10.000000}{12.000000}\selectfont\catcode`\^=\active\def^{\ifmmode\sp\else\^{}\fi}\catcode`\%=\active\def%{\%}$\mathdefault{\ensuremath{-}10}$}}%
\end{pgfscope}%
\begin{pgfscope}%
\pgfpathrectangle{\pgfqpoint{0.643596in}{0.515000in}}{\pgfqpoint{3.498815in}{2.525900in}}%
\pgfusepath{clip}%
\pgfsetrectcap%
\pgfsetroundjoin%
\pgfsetlinewidth{0.803000pt}%
\definecolor{currentstroke}{rgb}{0.690196,0.690196,0.690196}%
\pgfsetstrokecolor{currentstroke}%
\pgfsetdash{}{0pt}%
\pgfpathmoveto{\pgfqpoint{0.643596in}{1.777950in}}%
\pgfpathlineto{\pgfqpoint{4.142411in}{1.777950in}}%
\pgfusepath{stroke}%
\end{pgfscope}%
\begin{pgfscope}%
\pgfsetbuttcap%
\pgfsetroundjoin%
\definecolor{currentfill}{rgb}{0.000000,0.000000,0.000000}%
\pgfsetfillcolor{currentfill}%
\pgfsetlinewidth{0.803000pt}%
\definecolor{currentstroke}{rgb}{0.000000,0.000000,0.000000}%
\pgfsetstrokecolor{currentstroke}%
\pgfsetdash{}{0pt}%
\pgfsys@defobject{currentmarker}{\pgfqpoint{-0.048611in}{0.000000in}}{\pgfqpoint{-0.000000in}{0.000000in}}{%
\pgfpathmoveto{\pgfqpoint{-0.000000in}{0.000000in}}%
\pgfpathlineto{\pgfqpoint{-0.048611in}{0.000000in}}%
\pgfusepath{stroke,fill}%
}%
\begin{pgfscope}%
\pgfsys@transformshift{0.643596in}{1.777950in}%
\pgfsys@useobject{currentmarker}{}%
\end{pgfscope}%
\end{pgfscope}%
\begin{pgfscope}%
\definecolor{textcolor}{rgb}{0.000000,0.000000,0.000000}%
\pgfsetstrokecolor{textcolor}%
\pgfsetfillcolor{textcolor}%
\pgftext[x=0.476929in, y=1.729755in, left, base]{\color{textcolor}{\rmfamily\fontsize{10.000000}{12.000000}\selectfont\catcode`\^=\active\def^{\ifmmode\sp\else\^{}\fi}\catcode`\%=\active\def%{\%}$\mathdefault{0}$}}%
\end{pgfscope}%
\begin{pgfscope}%
\pgfpathrectangle{\pgfqpoint{0.643596in}{0.515000in}}{\pgfqpoint{3.498815in}{2.525900in}}%
\pgfusepath{clip}%
\pgfsetrectcap%
\pgfsetroundjoin%
\pgfsetlinewidth{0.803000pt}%
\definecolor{currentstroke}{rgb}{0.690196,0.690196,0.690196}%
\pgfsetstrokecolor{currentstroke}%
\pgfsetdash{}{0pt}%
\pgfpathmoveto{\pgfqpoint{0.643596in}{2.198933in}}%
\pgfpathlineto{\pgfqpoint{4.142411in}{2.198933in}}%
\pgfusepath{stroke}%
\end{pgfscope}%
\begin{pgfscope}%
\pgfsetbuttcap%
\pgfsetroundjoin%
\definecolor{currentfill}{rgb}{0.000000,0.000000,0.000000}%
\pgfsetfillcolor{currentfill}%
\pgfsetlinewidth{0.803000pt}%
\definecolor{currentstroke}{rgb}{0.000000,0.000000,0.000000}%
\pgfsetstrokecolor{currentstroke}%
\pgfsetdash{}{0pt}%
\pgfsys@defobject{currentmarker}{\pgfqpoint{-0.048611in}{0.000000in}}{\pgfqpoint{-0.000000in}{0.000000in}}{%
\pgfpathmoveto{\pgfqpoint{-0.000000in}{0.000000in}}%
\pgfpathlineto{\pgfqpoint{-0.048611in}{0.000000in}}%
\pgfusepath{stroke,fill}%
}%
\begin{pgfscope}%
\pgfsys@transformshift{0.643596in}{2.198933in}%
\pgfsys@useobject{currentmarker}{}%
\end{pgfscope}%
\end{pgfscope}%
\begin{pgfscope}%
\definecolor{textcolor}{rgb}{0.000000,0.000000,0.000000}%
\pgfsetstrokecolor{textcolor}%
\pgfsetfillcolor{textcolor}%
\pgftext[x=0.407484in, y=2.150739in, left, base]{\color{textcolor}{\rmfamily\fontsize{10.000000}{12.000000}\selectfont\catcode`\^=\active\def^{\ifmmode\sp\else\^{}\fi}\catcode`\%=\active\def%{\%}$\mathdefault{10}$}}%
\end{pgfscope}%
\begin{pgfscope}%
\pgfpathrectangle{\pgfqpoint{0.643596in}{0.515000in}}{\pgfqpoint{3.498815in}{2.525900in}}%
\pgfusepath{clip}%
\pgfsetrectcap%
\pgfsetroundjoin%
\pgfsetlinewidth{0.803000pt}%
\definecolor{currentstroke}{rgb}{0.690196,0.690196,0.690196}%
\pgfsetstrokecolor{currentstroke}%
\pgfsetdash{}{0pt}%
\pgfpathmoveto{\pgfqpoint{0.643596in}{2.619916in}}%
\pgfpathlineto{\pgfqpoint{4.142411in}{2.619916in}}%
\pgfusepath{stroke}%
\end{pgfscope}%
\begin{pgfscope}%
\pgfsetbuttcap%
\pgfsetroundjoin%
\definecolor{currentfill}{rgb}{0.000000,0.000000,0.000000}%
\pgfsetfillcolor{currentfill}%
\pgfsetlinewidth{0.803000pt}%
\definecolor{currentstroke}{rgb}{0.000000,0.000000,0.000000}%
\pgfsetstrokecolor{currentstroke}%
\pgfsetdash{}{0pt}%
\pgfsys@defobject{currentmarker}{\pgfqpoint{-0.048611in}{0.000000in}}{\pgfqpoint{-0.000000in}{0.000000in}}{%
\pgfpathmoveto{\pgfqpoint{-0.000000in}{0.000000in}}%
\pgfpathlineto{\pgfqpoint{-0.048611in}{0.000000in}}%
\pgfusepath{stroke,fill}%
}%
\begin{pgfscope}%
\pgfsys@transformshift{0.643596in}{2.619916in}%
\pgfsys@useobject{currentmarker}{}%
\end{pgfscope}%
\end{pgfscope}%
\begin{pgfscope}%
\definecolor{textcolor}{rgb}{0.000000,0.000000,0.000000}%
\pgfsetstrokecolor{textcolor}%
\pgfsetfillcolor{textcolor}%
\pgftext[x=0.407484in, y=2.571722in, left, base]{\color{textcolor}{\rmfamily\fontsize{10.000000}{12.000000}\selectfont\catcode`\^=\active\def^{\ifmmode\sp\else\^{}\fi}\catcode`\%=\active\def%{\%}$\mathdefault{20}$}}%
\end{pgfscope}%
\begin{pgfscope}%
\pgfpathrectangle{\pgfqpoint{0.643596in}{0.515000in}}{\pgfqpoint{3.498815in}{2.525900in}}%
\pgfusepath{clip}%
\pgfsetrectcap%
\pgfsetroundjoin%
\pgfsetlinewidth{0.803000pt}%
\definecolor{currentstroke}{rgb}{0.690196,0.690196,0.690196}%
\pgfsetstrokecolor{currentstroke}%
\pgfsetdash{}{0pt}%
\pgfpathmoveto{\pgfqpoint{0.643596in}{3.040900in}}%
\pgfpathlineto{\pgfqpoint{4.142411in}{3.040900in}}%
\pgfusepath{stroke}%
\end{pgfscope}%
\begin{pgfscope}%
\pgfsetbuttcap%
\pgfsetroundjoin%
\definecolor{currentfill}{rgb}{0.000000,0.000000,0.000000}%
\pgfsetfillcolor{currentfill}%
\pgfsetlinewidth{0.803000pt}%
\definecolor{currentstroke}{rgb}{0.000000,0.000000,0.000000}%
\pgfsetstrokecolor{currentstroke}%
\pgfsetdash{}{0pt}%
\pgfsys@defobject{currentmarker}{\pgfqpoint{-0.048611in}{0.000000in}}{\pgfqpoint{-0.000000in}{0.000000in}}{%
\pgfpathmoveto{\pgfqpoint{-0.000000in}{0.000000in}}%
\pgfpathlineto{\pgfqpoint{-0.048611in}{0.000000in}}%
\pgfusepath{stroke,fill}%
}%
\begin{pgfscope}%
\pgfsys@transformshift{0.643596in}{3.040900in}%
\pgfsys@useobject{currentmarker}{}%
\end{pgfscope}%
\end{pgfscope}%
\begin{pgfscope}%
\definecolor{textcolor}{rgb}{0.000000,0.000000,0.000000}%
\pgfsetstrokecolor{textcolor}%
\pgfsetfillcolor{textcolor}%
\pgftext[x=0.407484in, y=2.992705in, left, base]{\color{textcolor}{\rmfamily\fontsize{10.000000}{12.000000}\selectfont\catcode`\^=\active\def^{\ifmmode\sp\else\^{}\fi}\catcode`\%=\active\def%{\%}$\mathdefault{30}$}}%
\end{pgfscope}%
\begin{pgfscope}%
\pgfsetbuttcap%
\pgfsetroundjoin%
\definecolor{currentfill}{rgb}{0.000000,0.000000,0.000000}%
\pgfsetfillcolor{currentfill}%
\pgfsetlinewidth{0.602250pt}%
\definecolor{currentstroke}{rgb}{0.000000,0.000000,0.000000}%
\pgfsetstrokecolor{currentstroke}%
\pgfsetdash{}{0pt}%
\pgfsys@defobject{currentmarker}{\pgfqpoint{-0.027778in}{0.000000in}}{\pgfqpoint{-0.000000in}{0.000000in}}{%
\pgfpathmoveto{\pgfqpoint{-0.000000in}{0.000000in}}%
\pgfpathlineto{\pgfqpoint{-0.027778in}{0.000000in}}%
\pgfusepath{stroke,fill}%
}%
\begin{pgfscope}%
\pgfsys@transformshift{0.643596in}{0.599196in}%
\pgfsys@useobject{currentmarker}{}%
\end{pgfscope}%
\end{pgfscope}%
\begin{pgfscope}%
\pgfsetbuttcap%
\pgfsetroundjoin%
\definecolor{currentfill}{rgb}{0.000000,0.000000,0.000000}%
\pgfsetfillcolor{currentfill}%
\pgfsetlinewidth{0.602250pt}%
\definecolor{currentstroke}{rgb}{0.000000,0.000000,0.000000}%
\pgfsetstrokecolor{currentstroke}%
\pgfsetdash{}{0pt}%
\pgfsys@defobject{currentmarker}{\pgfqpoint{-0.027778in}{0.000000in}}{\pgfqpoint{-0.000000in}{0.000000in}}{%
\pgfpathmoveto{\pgfqpoint{-0.000000in}{0.000000in}}%
\pgfpathlineto{\pgfqpoint{-0.027778in}{0.000000in}}%
\pgfusepath{stroke,fill}%
}%
\begin{pgfscope}%
\pgfsys@transformshift{0.643596in}{0.683393in}%
\pgfsys@useobject{currentmarker}{}%
\end{pgfscope}%
\end{pgfscope}%
\begin{pgfscope}%
\pgfsetbuttcap%
\pgfsetroundjoin%
\definecolor{currentfill}{rgb}{0.000000,0.000000,0.000000}%
\pgfsetfillcolor{currentfill}%
\pgfsetlinewidth{0.602250pt}%
\definecolor{currentstroke}{rgb}{0.000000,0.000000,0.000000}%
\pgfsetstrokecolor{currentstroke}%
\pgfsetdash{}{0pt}%
\pgfsys@defobject{currentmarker}{\pgfqpoint{-0.027778in}{0.000000in}}{\pgfqpoint{-0.000000in}{0.000000in}}{%
\pgfpathmoveto{\pgfqpoint{-0.000000in}{0.000000in}}%
\pgfpathlineto{\pgfqpoint{-0.027778in}{0.000000in}}%
\pgfusepath{stroke,fill}%
}%
\begin{pgfscope}%
\pgfsys@transformshift{0.643596in}{0.767590in}%
\pgfsys@useobject{currentmarker}{}%
\end{pgfscope}%
\end{pgfscope}%
\begin{pgfscope}%
\pgfsetbuttcap%
\pgfsetroundjoin%
\definecolor{currentfill}{rgb}{0.000000,0.000000,0.000000}%
\pgfsetfillcolor{currentfill}%
\pgfsetlinewidth{0.602250pt}%
\definecolor{currentstroke}{rgb}{0.000000,0.000000,0.000000}%
\pgfsetstrokecolor{currentstroke}%
\pgfsetdash{}{0pt}%
\pgfsys@defobject{currentmarker}{\pgfqpoint{-0.027778in}{0.000000in}}{\pgfqpoint{-0.000000in}{0.000000in}}{%
\pgfpathmoveto{\pgfqpoint{-0.000000in}{0.000000in}}%
\pgfpathlineto{\pgfqpoint{-0.027778in}{0.000000in}}%
\pgfusepath{stroke,fill}%
}%
\begin{pgfscope}%
\pgfsys@transformshift{0.643596in}{0.851786in}%
\pgfsys@useobject{currentmarker}{}%
\end{pgfscope}%
\end{pgfscope}%
\begin{pgfscope}%
\pgfsetbuttcap%
\pgfsetroundjoin%
\definecolor{currentfill}{rgb}{0.000000,0.000000,0.000000}%
\pgfsetfillcolor{currentfill}%
\pgfsetlinewidth{0.602250pt}%
\definecolor{currentstroke}{rgb}{0.000000,0.000000,0.000000}%
\pgfsetstrokecolor{currentstroke}%
\pgfsetdash{}{0pt}%
\pgfsys@defobject{currentmarker}{\pgfqpoint{-0.027778in}{0.000000in}}{\pgfqpoint{-0.000000in}{0.000000in}}{%
\pgfpathmoveto{\pgfqpoint{-0.000000in}{0.000000in}}%
\pgfpathlineto{\pgfqpoint{-0.027778in}{0.000000in}}%
\pgfusepath{stroke,fill}%
}%
\begin{pgfscope}%
\pgfsys@transformshift{0.643596in}{1.020180in}%
\pgfsys@useobject{currentmarker}{}%
\end{pgfscope}%
\end{pgfscope}%
\begin{pgfscope}%
\pgfsetbuttcap%
\pgfsetroundjoin%
\definecolor{currentfill}{rgb}{0.000000,0.000000,0.000000}%
\pgfsetfillcolor{currentfill}%
\pgfsetlinewidth{0.602250pt}%
\definecolor{currentstroke}{rgb}{0.000000,0.000000,0.000000}%
\pgfsetstrokecolor{currentstroke}%
\pgfsetdash{}{0pt}%
\pgfsys@defobject{currentmarker}{\pgfqpoint{-0.027778in}{0.000000in}}{\pgfqpoint{-0.000000in}{0.000000in}}{%
\pgfpathmoveto{\pgfqpoint{-0.000000in}{0.000000in}}%
\pgfpathlineto{\pgfqpoint{-0.027778in}{0.000000in}}%
\pgfusepath{stroke,fill}%
}%
\begin{pgfscope}%
\pgfsys@transformshift{0.643596in}{1.104376in}%
\pgfsys@useobject{currentmarker}{}%
\end{pgfscope}%
\end{pgfscope}%
\begin{pgfscope}%
\pgfsetbuttcap%
\pgfsetroundjoin%
\definecolor{currentfill}{rgb}{0.000000,0.000000,0.000000}%
\pgfsetfillcolor{currentfill}%
\pgfsetlinewidth{0.602250pt}%
\definecolor{currentstroke}{rgb}{0.000000,0.000000,0.000000}%
\pgfsetstrokecolor{currentstroke}%
\pgfsetdash{}{0pt}%
\pgfsys@defobject{currentmarker}{\pgfqpoint{-0.027778in}{0.000000in}}{\pgfqpoint{-0.000000in}{0.000000in}}{%
\pgfpathmoveto{\pgfqpoint{-0.000000in}{0.000000in}}%
\pgfpathlineto{\pgfqpoint{-0.027778in}{0.000000in}}%
\pgfusepath{stroke,fill}%
}%
\begin{pgfscope}%
\pgfsys@transformshift{0.643596in}{1.188573in}%
\pgfsys@useobject{currentmarker}{}%
\end{pgfscope}%
\end{pgfscope}%
\begin{pgfscope}%
\pgfsetbuttcap%
\pgfsetroundjoin%
\definecolor{currentfill}{rgb}{0.000000,0.000000,0.000000}%
\pgfsetfillcolor{currentfill}%
\pgfsetlinewidth{0.602250pt}%
\definecolor{currentstroke}{rgb}{0.000000,0.000000,0.000000}%
\pgfsetstrokecolor{currentstroke}%
\pgfsetdash{}{0pt}%
\pgfsys@defobject{currentmarker}{\pgfqpoint{-0.027778in}{0.000000in}}{\pgfqpoint{-0.000000in}{0.000000in}}{%
\pgfpathmoveto{\pgfqpoint{-0.000000in}{0.000000in}}%
\pgfpathlineto{\pgfqpoint{-0.027778in}{0.000000in}}%
\pgfusepath{stroke,fill}%
}%
\begin{pgfscope}%
\pgfsys@transformshift{0.643596in}{1.272770in}%
\pgfsys@useobject{currentmarker}{}%
\end{pgfscope}%
\end{pgfscope}%
\begin{pgfscope}%
\pgfsetbuttcap%
\pgfsetroundjoin%
\definecolor{currentfill}{rgb}{0.000000,0.000000,0.000000}%
\pgfsetfillcolor{currentfill}%
\pgfsetlinewidth{0.602250pt}%
\definecolor{currentstroke}{rgb}{0.000000,0.000000,0.000000}%
\pgfsetstrokecolor{currentstroke}%
\pgfsetdash{}{0pt}%
\pgfsys@defobject{currentmarker}{\pgfqpoint{-0.027778in}{0.000000in}}{\pgfqpoint{-0.000000in}{0.000000in}}{%
\pgfpathmoveto{\pgfqpoint{-0.000000in}{0.000000in}}%
\pgfpathlineto{\pgfqpoint{-0.027778in}{0.000000in}}%
\pgfusepath{stroke,fill}%
}%
\begin{pgfscope}%
\pgfsys@transformshift{0.643596in}{1.441163in}%
\pgfsys@useobject{currentmarker}{}%
\end{pgfscope}%
\end{pgfscope}%
\begin{pgfscope}%
\pgfsetbuttcap%
\pgfsetroundjoin%
\definecolor{currentfill}{rgb}{0.000000,0.000000,0.000000}%
\pgfsetfillcolor{currentfill}%
\pgfsetlinewidth{0.602250pt}%
\definecolor{currentstroke}{rgb}{0.000000,0.000000,0.000000}%
\pgfsetstrokecolor{currentstroke}%
\pgfsetdash{}{0pt}%
\pgfsys@defobject{currentmarker}{\pgfqpoint{-0.027778in}{0.000000in}}{\pgfqpoint{-0.000000in}{0.000000in}}{%
\pgfpathmoveto{\pgfqpoint{-0.000000in}{0.000000in}}%
\pgfpathlineto{\pgfqpoint{-0.027778in}{0.000000in}}%
\pgfusepath{stroke,fill}%
}%
\begin{pgfscope}%
\pgfsys@transformshift{0.643596in}{1.525360in}%
\pgfsys@useobject{currentmarker}{}%
\end{pgfscope}%
\end{pgfscope}%
\begin{pgfscope}%
\pgfsetbuttcap%
\pgfsetroundjoin%
\definecolor{currentfill}{rgb}{0.000000,0.000000,0.000000}%
\pgfsetfillcolor{currentfill}%
\pgfsetlinewidth{0.602250pt}%
\definecolor{currentstroke}{rgb}{0.000000,0.000000,0.000000}%
\pgfsetstrokecolor{currentstroke}%
\pgfsetdash{}{0pt}%
\pgfsys@defobject{currentmarker}{\pgfqpoint{-0.027778in}{0.000000in}}{\pgfqpoint{-0.000000in}{0.000000in}}{%
\pgfpathmoveto{\pgfqpoint{-0.000000in}{0.000000in}}%
\pgfpathlineto{\pgfqpoint{-0.027778in}{0.000000in}}%
\pgfusepath{stroke,fill}%
}%
\begin{pgfscope}%
\pgfsys@transformshift{0.643596in}{1.609556in}%
\pgfsys@useobject{currentmarker}{}%
\end{pgfscope}%
\end{pgfscope}%
\begin{pgfscope}%
\pgfsetbuttcap%
\pgfsetroundjoin%
\definecolor{currentfill}{rgb}{0.000000,0.000000,0.000000}%
\pgfsetfillcolor{currentfill}%
\pgfsetlinewidth{0.602250pt}%
\definecolor{currentstroke}{rgb}{0.000000,0.000000,0.000000}%
\pgfsetstrokecolor{currentstroke}%
\pgfsetdash{}{0pt}%
\pgfsys@defobject{currentmarker}{\pgfqpoint{-0.027778in}{0.000000in}}{\pgfqpoint{-0.000000in}{0.000000in}}{%
\pgfpathmoveto{\pgfqpoint{-0.000000in}{0.000000in}}%
\pgfpathlineto{\pgfqpoint{-0.027778in}{0.000000in}}%
\pgfusepath{stroke,fill}%
}%
\begin{pgfscope}%
\pgfsys@transformshift{0.643596in}{1.693753in}%
\pgfsys@useobject{currentmarker}{}%
\end{pgfscope}%
\end{pgfscope}%
\begin{pgfscope}%
\pgfsetbuttcap%
\pgfsetroundjoin%
\definecolor{currentfill}{rgb}{0.000000,0.000000,0.000000}%
\pgfsetfillcolor{currentfill}%
\pgfsetlinewidth{0.602250pt}%
\definecolor{currentstroke}{rgb}{0.000000,0.000000,0.000000}%
\pgfsetstrokecolor{currentstroke}%
\pgfsetdash{}{0pt}%
\pgfsys@defobject{currentmarker}{\pgfqpoint{-0.027778in}{0.000000in}}{\pgfqpoint{-0.000000in}{0.000000in}}{%
\pgfpathmoveto{\pgfqpoint{-0.000000in}{0.000000in}}%
\pgfpathlineto{\pgfqpoint{-0.027778in}{0.000000in}}%
\pgfusepath{stroke,fill}%
}%
\begin{pgfscope}%
\pgfsys@transformshift{0.643596in}{1.862146in}%
\pgfsys@useobject{currentmarker}{}%
\end{pgfscope}%
\end{pgfscope}%
\begin{pgfscope}%
\pgfsetbuttcap%
\pgfsetroundjoin%
\definecolor{currentfill}{rgb}{0.000000,0.000000,0.000000}%
\pgfsetfillcolor{currentfill}%
\pgfsetlinewidth{0.602250pt}%
\definecolor{currentstroke}{rgb}{0.000000,0.000000,0.000000}%
\pgfsetstrokecolor{currentstroke}%
\pgfsetdash{}{0pt}%
\pgfsys@defobject{currentmarker}{\pgfqpoint{-0.027778in}{0.000000in}}{\pgfqpoint{-0.000000in}{0.000000in}}{%
\pgfpathmoveto{\pgfqpoint{-0.000000in}{0.000000in}}%
\pgfpathlineto{\pgfqpoint{-0.027778in}{0.000000in}}%
\pgfusepath{stroke,fill}%
}%
\begin{pgfscope}%
\pgfsys@transformshift{0.643596in}{1.946343in}%
\pgfsys@useobject{currentmarker}{}%
\end{pgfscope}%
\end{pgfscope}%
\begin{pgfscope}%
\pgfsetbuttcap%
\pgfsetroundjoin%
\definecolor{currentfill}{rgb}{0.000000,0.000000,0.000000}%
\pgfsetfillcolor{currentfill}%
\pgfsetlinewidth{0.602250pt}%
\definecolor{currentstroke}{rgb}{0.000000,0.000000,0.000000}%
\pgfsetstrokecolor{currentstroke}%
\pgfsetdash{}{0pt}%
\pgfsys@defobject{currentmarker}{\pgfqpoint{-0.027778in}{0.000000in}}{\pgfqpoint{-0.000000in}{0.000000in}}{%
\pgfpathmoveto{\pgfqpoint{-0.000000in}{0.000000in}}%
\pgfpathlineto{\pgfqpoint{-0.027778in}{0.000000in}}%
\pgfusepath{stroke,fill}%
}%
\begin{pgfscope}%
\pgfsys@transformshift{0.643596in}{2.030540in}%
\pgfsys@useobject{currentmarker}{}%
\end{pgfscope}%
\end{pgfscope}%
\begin{pgfscope}%
\pgfsetbuttcap%
\pgfsetroundjoin%
\definecolor{currentfill}{rgb}{0.000000,0.000000,0.000000}%
\pgfsetfillcolor{currentfill}%
\pgfsetlinewidth{0.602250pt}%
\definecolor{currentstroke}{rgb}{0.000000,0.000000,0.000000}%
\pgfsetstrokecolor{currentstroke}%
\pgfsetdash{}{0pt}%
\pgfsys@defobject{currentmarker}{\pgfqpoint{-0.027778in}{0.000000in}}{\pgfqpoint{-0.000000in}{0.000000in}}{%
\pgfpathmoveto{\pgfqpoint{-0.000000in}{0.000000in}}%
\pgfpathlineto{\pgfqpoint{-0.027778in}{0.000000in}}%
\pgfusepath{stroke,fill}%
}%
\begin{pgfscope}%
\pgfsys@transformshift{0.643596in}{2.114736in}%
\pgfsys@useobject{currentmarker}{}%
\end{pgfscope}%
\end{pgfscope}%
\begin{pgfscope}%
\pgfsetbuttcap%
\pgfsetroundjoin%
\definecolor{currentfill}{rgb}{0.000000,0.000000,0.000000}%
\pgfsetfillcolor{currentfill}%
\pgfsetlinewidth{0.602250pt}%
\definecolor{currentstroke}{rgb}{0.000000,0.000000,0.000000}%
\pgfsetstrokecolor{currentstroke}%
\pgfsetdash{}{0pt}%
\pgfsys@defobject{currentmarker}{\pgfqpoint{-0.027778in}{0.000000in}}{\pgfqpoint{-0.000000in}{0.000000in}}{%
\pgfpathmoveto{\pgfqpoint{-0.000000in}{0.000000in}}%
\pgfpathlineto{\pgfqpoint{-0.027778in}{0.000000in}}%
\pgfusepath{stroke,fill}%
}%
\begin{pgfscope}%
\pgfsys@transformshift{0.643596in}{2.283130in}%
\pgfsys@useobject{currentmarker}{}%
\end{pgfscope}%
\end{pgfscope}%
\begin{pgfscope}%
\pgfsetbuttcap%
\pgfsetroundjoin%
\definecolor{currentfill}{rgb}{0.000000,0.000000,0.000000}%
\pgfsetfillcolor{currentfill}%
\pgfsetlinewidth{0.602250pt}%
\definecolor{currentstroke}{rgb}{0.000000,0.000000,0.000000}%
\pgfsetstrokecolor{currentstroke}%
\pgfsetdash{}{0pt}%
\pgfsys@defobject{currentmarker}{\pgfqpoint{-0.027778in}{0.000000in}}{\pgfqpoint{-0.000000in}{0.000000in}}{%
\pgfpathmoveto{\pgfqpoint{-0.000000in}{0.000000in}}%
\pgfpathlineto{\pgfqpoint{-0.027778in}{0.000000in}}%
\pgfusepath{stroke,fill}%
}%
\begin{pgfscope}%
\pgfsys@transformshift{0.643596in}{2.367326in}%
\pgfsys@useobject{currentmarker}{}%
\end{pgfscope}%
\end{pgfscope}%
\begin{pgfscope}%
\pgfsetbuttcap%
\pgfsetroundjoin%
\definecolor{currentfill}{rgb}{0.000000,0.000000,0.000000}%
\pgfsetfillcolor{currentfill}%
\pgfsetlinewidth{0.602250pt}%
\definecolor{currentstroke}{rgb}{0.000000,0.000000,0.000000}%
\pgfsetstrokecolor{currentstroke}%
\pgfsetdash{}{0pt}%
\pgfsys@defobject{currentmarker}{\pgfqpoint{-0.027778in}{0.000000in}}{\pgfqpoint{-0.000000in}{0.000000in}}{%
\pgfpathmoveto{\pgfqpoint{-0.000000in}{0.000000in}}%
\pgfpathlineto{\pgfqpoint{-0.027778in}{0.000000in}}%
\pgfusepath{stroke,fill}%
}%
\begin{pgfscope}%
\pgfsys@transformshift{0.643596in}{2.451523in}%
\pgfsys@useobject{currentmarker}{}%
\end{pgfscope}%
\end{pgfscope}%
\begin{pgfscope}%
\pgfsetbuttcap%
\pgfsetroundjoin%
\definecolor{currentfill}{rgb}{0.000000,0.000000,0.000000}%
\pgfsetfillcolor{currentfill}%
\pgfsetlinewidth{0.602250pt}%
\definecolor{currentstroke}{rgb}{0.000000,0.000000,0.000000}%
\pgfsetstrokecolor{currentstroke}%
\pgfsetdash{}{0pt}%
\pgfsys@defobject{currentmarker}{\pgfqpoint{-0.027778in}{0.000000in}}{\pgfqpoint{-0.000000in}{0.000000in}}{%
\pgfpathmoveto{\pgfqpoint{-0.000000in}{0.000000in}}%
\pgfpathlineto{\pgfqpoint{-0.027778in}{0.000000in}}%
\pgfusepath{stroke,fill}%
}%
\begin{pgfscope}%
\pgfsys@transformshift{0.643596in}{2.535720in}%
\pgfsys@useobject{currentmarker}{}%
\end{pgfscope}%
\end{pgfscope}%
\begin{pgfscope}%
\pgfsetbuttcap%
\pgfsetroundjoin%
\definecolor{currentfill}{rgb}{0.000000,0.000000,0.000000}%
\pgfsetfillcolor{currentfill}%
\pgfsetlinewidth{0.602250pt}%
\definecolor{currentstroke}{rgb}{0.000000,0.000000,0.000000}%
\pgfsetstrokecolor{currentstroke}%
\pgfsetdash{}{0pt}%
\pgfsys@defobject{currentmarker}{\pgfqpoint{-0.027778in}{0.000000in}}{\pgfqpoint{-0.000000in}{0.000000in}}{%
\pgfpathmoveto{\pgfqpoint{-0.000000in}{0.000000in}}%
\pgfpathlineto{\pgfqpoint{-0.027778in}{0.000000in}}%
\pgfusepath{stroke,fill}%
}%
\begin{pgfscope}%
\pgfsys@transformshift{0.643596in}{2.704113in}%
\pgfsys@useobject{currentmarker}{}%
\end{pgfscope}%
\end{pgfscope}%
\begin{pgfscope}%
\pgfsetbuttcap%
\pgfsetroundjoin%
\definecolor{currentfill}{rgb}{0.000000,0.000000,0.000000}%
\pgfsetfillcolor{currentfill}%
\pgfsetlinewidth{0.602250pt}%
\definecolor{currentstroke}{rgb}{0.000000,0.000000,0.000000}%
\pgfsetstrokecolor{currentstroke}%
\pgfsetdash{}{0pt}%
\pgfsys@defobject{currentmarker}{\pgfqpoint{-0.027778in}{0.000000in}}{\pgfqpoint{-0.000000in}{0.000000in}}{%
\pgfpathmoveto{\pgfqpoint{-0.000000in}{0.000000in}}%
\pgfpathlineto{\pgfqpoint{-0.027778in}{0.000000in}}%
\pgfusepath{stroke,fill}%
}%
\begin{pgfscope}%
\pgfsys@transformshift{0.643596in}{2.788310in}%
\pgfsys@useobject{currentmarker}{}%
\end{pgfscope}%
\end{pgfscope}%
\begin{pgfscope}%
\pgfsetbuttcap%
\pgfsetroundjoin%
\definecolor{currentfill}{rgb}{0.000000,0.000000,0.000000}%
\pgfsetfillcolor{currentfill}%
\pgfsetlinewidth{0.602250pt}%
\definecolor{currentstroke}{rgb}{0.000000,0.000000,0.000000}%
\pgfsetstrokecolor{currentstroke}%
\pgfsetdash{}{0pt}%
\pgfsys@defobject{currentmarker}{\pgfqpoint{-0.027778in}{0.000000in}}{\pgfqpoint{-0.000000in}{0.000000in}}{%
\pgfpathmoveto{\pgfqpoint{-0.000000in}{0.000000in}}%
\pgfpathlineto{\pgfqpoint{-0.027778in}{0.000000in}}%
\pgfusepath{stroke,fill}%
}%
\begin{pgfscope}%
\pgfsys@transformshift{0.643596in}{2.872507in}%
\pgfsys@useobject{currentmarker}{}%
\end{pgfscope}%
\end{pgfscope}%
\begin{pgfscope}%
\pgfsetbuttcap%
\pgfsetroundjoin%
\definecolor{currentfill}{rgb}{0.000000,0.000000,0.000000}%
\pgfsetfillcolor{currentfill}%
\pgfsetlinewidth{0.602250pt}%
\definecolor{currentstroke}{rgb}{0.000000,0.000000,0.000000}%
\pgfsetstrokecolor{currentstroke}%
\pgfsetdash{}{0pt}%
\pgfsys@defobject{currentmarker}{\pgfqpoint{-0.027778in}{0.000000in}}{\pgfqpoint{-0.000000in}{0.000000in}}{%
\pgfpathmoveto{\pgfqpoint{-0.000000in}{0.000000in}}%
\pgfpathlineto{\pgfqpoint{-0.027778in}{0.000000in}}%
\pgfusepath{stroke,fill}%
}%
\begin{pgfscope}%
\pgfsys@transformshift{0.643596in}{2.956703in}%
\pgfsys@useobject{currentmarker}{}%
\end{pgfscope}%
\end{pgfscope}%
\begin{pgfscope}%
\definecolor{textcolor}{rgb}{0.000000,0.000000,0.000000}%
\pgfsetstrokecolor{textcolor}%
\pgfsetfillcolor{textcolor}%
\pgftext[x=0.243904in,y=1.777950in,,bottom,rotate=90.000000]{\color{textcolor}{\rmfamily\fontsize{10.000000}{12.000000}\selectfont\catcode`\^=\active\def^{\ifmmode\sp\else\^{}\fi}\catcode`\%=\active\def%{\%}$S_{ij}$ (dB)}}%
\end{pgfscope}%
\begin{pgfscope}%
\pgfpathrectangle{\pgfqpoint{0.643596in}{0.515000in}}{\pgfqpoint{3.498815in}{2.525900in}}%
\pgfusepath{clip}%
\pgfsetrectcap%
\pgfsetroundjoin%
\pgfsetlinewidth{1.505625pt}%
\definecolor{currentstroke}{rgb}{0.000000,0.000000,1.000000}%
\pgfsetstrokecolor{currentstroke}%
\pgfsetstrokeopacity{0.700000}%
\pgfsetdash{}{0pt}%
\pgfpathmoveto{\pgfqpoint{0.647095in}{2.585467in}}%
\pgfpathlineto{\pgfqpoint{0.682065in}{2.691888in}}%
\pgfpathlineto{\pgfqpoint{0.717036in}{2.674001in}}%
\pgfpathlineto{\pgfqpoint{0.752007in}{2.719059in}}%
\pgfpathlineto{\pgfqpoint{0.786977in}{2.705229in}}%
\pgfpathlineto{\pgfqpoint{0.821948in}{2.715737in}}%
\pgfpathlineto{\pgfqpoint{0.856919in}{2.708456in}}%
\pgfpathlineto{\pgfqpoint{0.926860in}{2.705565in}}%
\pgfpathlineto{\pgfqpoint{0.961831in}{2.711236in}}%
\pgfpathlineto{\pgfqpoint{0.996801in}{2.701658in}}%
\pgfpathlineto{\pgfqpoint{1.031772in}{2.682617in}}%
\pgfpathlineto{\pgfqpoint{1.066743in}{2.694825in}}%
\pgfpathlineto{\pgfqpoint{1.101713in}{2.700321in}}%
\pgfpathlineto{\pgfqpoint{1.136684in}{2.692868in}}%
\pgfpathlineto{\pgfqpoint{1.171655in}{2.695498in}}%
\pgfpathlineto{\pgfqpoint{1.206625in}{2.691058in}}%
\pgfpathlineto{\pgfqpoint{1.241596in}{2.679129in}}%
\pgfpathlineto{\pgfqpoint{1.276567in}{2.669812in}}%
\pgfpathlineto{\pgfqpoint{1.311537in}{2.674920in}}%
\pgfpathlineto{\pgfqpoint{1.346508in}{2.682937in}}%
\pgfpathlineto{\pgfqpoint{1.381479in}{2.665016in}}%
\pgfpathlineto{\pgfqpoint{1.416449in}{2.682251in}}%
\pgfpathlineto{\pgfqpoint{1.451420in}{2.676706in}}%
\pgfpathlineto{\pgfqpoint{1.486391in}{2.693318in}}%
\pgfpathlineto{\pgfqpoint{1.521361in}{2.687323in}}%
\pgfpathlineto{\pgfqpoint{1.556332in}{2.696157in}}%
\pgfpathlineto{\pgfqpoint{1.591303in}{2.696166in}}%
\pgfpathlineto{\pgfqpoint{1.626273in}{2.690351in}}%
\pgfpathlineto{\pgfqpoint{1.661244in}{2.686431in}}%
\pgfpathlineto{\pgfqpoint{1.696214in}{2.685080in}}%
\pgfpathlineto{\pgfqpoint{1.731185in}{2.679019in}}%
\pgfpathlineto{\pgfqpoint{1.766156in}{2.677840in}}%
\pgfpathlineto{\pgfqpoint{1.801126in}{2.680615in}}%
\pgfpathlineto{\pgfqpoint{1.836097in}{2.681459in}}%
\pgfpathlineto{\pgfqpoint{1.871068in}{2.675016in}}%
\pgfpathlineto{\pgfqpoint{1.941009in}{2.669349in}}%
\pgfpathlineto{\pgfqpoint{1.975980in}{2.676322in}}%
\pgfpathlineto{\pgfqpoint{2.010950in}{2.681034in}}%
\pgfpathlineto{\pgfqpoint{2.045921in}{2.681912in}}%
\pgfpathlineto{\pgfqpoint{2.080892in}{2.688111in}}%
\pgfpathlineto{\pgfqpoint{2.115862in}{2.683301in}}%
\pgfpathlineto{\pgfqpoint{2.150833in}{2.687160in}}%
\pgfpathlineto{\pgfqpoint{2.220774in}{2.684854in}}%
\pgfpathlineto{\pgfqpoint{2.255745in}{2.666143in}}%
\pgfpathlineto{\pgfqpoint{2.290716in}{2.665954in}}%
\pgfpathlineto{\pgfqpoint{2.325686in}{2.675522in}}%
\pgfpathlineto{\pgfqpoint{2.360657in}{2.670277in}}%
\pgfpathlineto{\pgfqpoint{2.395628in}{2.660882in}}%
\pgfpathlineto{\pgfqpoint{2.430598in}{2.663261in}}%
\pgfpathlineto{\pgfqpoint{2.465569in}{2.649010in}}%
\pgfpathlineto{\pgfqpoint{2.500540in}{2.653169in}}%
\pgfpathlineto{\pgfqpoint{2.535510in}{2.646651in}}%
\pgfpathlineto{\pgfqpoint{2.570481in}{2.629035in}}%
\pgfpathlineto{\pgfqpoint{2.605452in}{2.638359in}}%
\pgfpathlineto{\pgfqpoint{2.640422in}{2.633402in}}%
\pgfpathlineto{\pgfqpoint{2.675393in}{2.643399in}}%
\pgfpathlineto{\pgfqpoint{2.710364in}{2.634407in}}%
\pgfpathlineto{\pgfqpoint{2.745334in}{2.645017in}}%
\pgfpathlineto{\pgfqpoint{2.780305in}{2.621890in}}%
\pgfpathlineto{\pgfqpoint{2.815275in}{2.627407in}}%
\pgfpathlineto{\pgfqpoint{2.850246in}{2.624921in}}%
\pgfpathlineto{\pgfqpoint{2.885217in}{2.645642in}}%
\pgfpathlineto{\pgfqpoint{2.920187in}{2.630990in}}%
\pgfpathlineto{\pgfqpoint{2.955158in}{2.639062in}}%
\pgfpathlineto{\pgfqpoint{2.990129in}{2.636735in}}%
\pgfpathlineto{\pgfqpoint{3.025099in}{2.632472in}}%
\pgfpathlineto{\pgfqpoint{3.060070in}{2.632298in}}%
\pgfpathlineto{\pgfqpoint{3.095041in}{2.616860in}}%
\pgfpathlineto{\pgfqpoint{3.130011in}{2.611691in}}%
\pgfpathlineto{\pgfqpoint{3.164982in}{2.611209in}}%
\pgfpathlineto{\pgfqpoint{3.199953in}{2.612178in}}%
\pgfpathlineto{\pgfqpoint{3.234923in}{2.590787in}}%
\pgfpathlineto{\pgfqpoint{3.269894in}{2.581702in}}%
\pgfpathlineto{\pgfqpoint{3.304865in}{2.579971in}}%
\pgfpathlineto{\pgfqpoint{3.339835in}{2.579565in}}%
\pgfpathlineto{\pgfqpoint{3.374806in}{2.601856in}}%
\pgfpathlineto{\pgfqpoint{3.409777in}{2.580830in}}%
\pgfpathlineto{\pgfqpoint{3.444747in}{2.623418in}}%
\pgfpathlineto{\pgfqpoint{3.514689in}{2.629201in}}%
\pgfpathlineto{\pgfqpoint{3.549659in}{2.616370in}}%
\pgfpathlineto{\pgfqpoint{3.584630in}{2.622534in}}%
\pgfpathlineto{\pgfqpoint{3.619601in}{2.604735in}}%
\pgfpathlineto{\pgfqpoint{3.654571in}{2.613582in}}%
\pgfpathlineto{\pgfqpoint{3.689542in}{2.601149in}}%
\pgfpathlineto{\pgfqpoint{3.724513in}{2.595510in}}%
\pgfpathlineto{\pgfqpoint{3.759483in}{2.599037in}}%
\pgfpathlineto{\pgfqpoint{3.794454in}{2.596453in}}%
\pgfpathlineto{\pgfqpoint{3.829425in}{2.591698in}}%
\pgfpathlineto{\pgfqpoint{3.864395in}{2.594508in}}%
\pgfpathlineto{\pgfqpoint{3.899366in}{2.594691in}}%
\pgfpathlineto{\pgfqpoint{3.934336in}{2.590104in}}%
\pgfpathlineto{\pgfqpoint{3.969307in}{2.562970in}}%
\pgfpathlineto{\pgfqpoint{4.004278in}{2.560102in}}%
\pgfpathlineto{\pgfqpoint{4.039248in}{2.554242in}}%
\pgfpathlineto{\pgfqpoint{4.074219in}{2.555947in}}%
\pgfpathlineto{\pgfqpoint{4.144160in}{2.543894in}}%
\pgfpathlineto{\pgfqpoint{4.152411in}{2.539076in}}%
\pgfpathlineto{\pgfqpoint{4.152411in}{2.539076in}}%
\pgfusepath{stroke}%
\end{pgfscope}%
\begin{pgfscope}%
\pgfpathrectangle{\pgfqpoint{0.643596in}{0.515000in}}{\pgfqpoint{3.498815in}{2.525900in}}%
\pgfusepath{clip}%
\pgfsetrectcap%
\pgfsetroundjoin%
\pgfsetlinewidth{1.505625pt}%
\definecolor{currentstroke}{rgb}{1.000000,0.647059,0.000000}%
\pgfsetstrokecolor{currentstroke}%
\pgfsetstrokeopacity{0.700000}%
\pgfsetdash{}{0pt}%
\pgfpathmoveto{\pgfqpoint{0.647095in}{1.285488in}}%
\pgfpathlineto{\pgfqpoint{0.682065in}{1.099151in}}%
\pgfpathlineto{\pgfqpoint{0.717036in}{0.803924in}}%
\pgfpathlineto{\pgfqpoint{0.752007in}{0.949168in}}%
\pgfpathlineto{\pgfqpoint{0.786977in}{0.872708in}}%
\pgfpathlineto{\pgfqpoint{0.821948in}{0.935405in}}%
\pgfpathlineto{\pgfqpoint{0.856919in}{0.803668in}}%
\pgfpathlineto{\pgfqpoint{0.891889in}{1.065081in}}%
\pgfpathlineto{\pgfqpoint{0.926860in}{0.956452in}}%
\pgfpathlineto{\pgfqpoint{0.961831in}{1.236308in}}%
\pgfpathlineto{\pgfqpoint{0.996801in}{0.984625in}}%
\pgfpathlineto{\pgfqpoint{1.031772in}{1.175671in}}%
\pgfpathlineto{\pgfqpoint{1.066743in}{1.177423in}}%
\pgfpathlineto{\pgfqpoint{1.101713in}{1.290094in}}%
\pgfpathlineto{\pgfqpoint{1.136684in}{1.257003in}}%
\pgfpathlineto{\pgfqpoint{1.171655in}{1.254333in}}%
\pgfpathlineto{\pgfqpoint{1.206625in}{1.286699in}}%
\pgfpathlineto{\pgfqpoint{1.241596in}{1.228593in}}%
\pgfpathlineto{\pgfqpoint{1.276567in}{1.239582in}}%
\pgfpathlineto{\pgfqpoint{1.311537in}{1.241089in}}%
\pgfpathlineto{\pgfqpoint{1.346508in}{1.257357in}}%
\pgfpathlineto{\pgfqpoint{1.381479in}{1.219171in}}%
\pgfpathlineto{\pgfqpoint{1.416449in}{1.241347in}}%
\pgfpathlineto{\pgfqpoint{1.451420in}{1.267989in}}%
\pgfpathlineto{\pgfqpoint{1.486391in}{1.198303in}}%
\pgfpathlineto{\pgfqpoint{1.521361in}{1.061277in}}%
\pgfpathlineto{\pgfqpoint{1.556332in}{0.756082in}}%
\pgfpathlineto{\pgfqpoint{1.591303in}{0.889900in}}%
\pgfpathlineto{\pgfqpoint{1.626273in}{0.791628in}}%
\pgfpathlineto{\pgfqpoint{1.661244in}{0.992366in}}%
\pgfpathlineto{\pgfqpoint{1.696214in}{0.847977in}}%
\pgfpathlineto{\pgfqpoint{1.731185in}{1.062809in}}%
\pgfpathlineto{\pgfqpoint{1.766156in}{1.038599in}}%
\pgfpathlineto{\pgfqpoint{1.801126in}{1.061276in}}%
\pgfpathlineto{\pgfqpoint{1.836097in}{0.840159in}}%
\pgfpathlineto{\pgfqpoint{1.871068in}{0.876744in}}%
\pgfpathlineto{\pgfqpoint{1.906038in}{1.117219in}}%
\pgfpathlineto{\pgfqpoint{1.941009in}{0.869034in}}%
\pgfpathlineto{\pgfqpoint{1.975980in}{0.950238in}}%
\pgfpathlineto{\pgfqpoint{2.004778in}{0.505000in}}%
\pgfpathmoveto{\pgfqpoint{2.016605in}{0.505000in}}%
\pgfpathlineto{\pgfqpoint{2.045921in}{0.999807in}}%
\pgfpathlineto{\pgfqpoint{2.080892in}{0.757099in}}%
\pgfpathlineto{\pgfqpoint{2.115862in}{0.719736in}}%
\pgfpathlineto{\pgfqpoint{2.150833in}{0.916704in}}%
\pgfpathlineto{\pgfqpoint{2.185804in}{1.136470in}}%
\pgfpathlineto{\pgfqpoint{2.220774in}{1.064837in}}%
\pgfpathlineto{\pgfqpoint{2.255745in}{0.585838in}}%
\pgfpathlineto{\pgfqpoint{2.290716in}{0.858998in}}%
\pgfpathlineto{\pgfqpoint{2.319474in}{0.505000in}}%
\pgfpathmoveto{\pgfqpoint{2.333991in}{0.505000in}}%
\pgfpathlineto{\pgfqpoint{2.395628in}{1.069378in}}%
\pgfpathlineto{\pgfqpoint{2.430598in}{1.126065in}}%
\pgfpathlineto{\pgfqpoint{2.465569in}{1.239031in}}%
\pgfpathlineto{\pgfqpoint{2.500540in}{1.338110in}}%
\pgfpathlineto{\pgfqpoint{2.535510in}{1.320054in}}%
\pgfpathlineto{\pgfqpoint{2.570481in}{1.436760in}}%
\pgfpathlineto{\pgfqpoint{2.605452in}{1.375183in}}%
\pgfpathlineto{\pgfqpoint{2.640422in}{1.471088in}}%
\pgfpathlineto{\pgfqpoint{2.675393in}{1.432660in}}%
\pgfpathlineto{\pgfqpoint{2.710364in}{1.481196in}}%
\pgfpathlineto{\pgfqpoint{2.745334in}{1.415977in}}%
\pgfpathlineto{\pgfqpoint{2.780305in}{1.461944in}}%
\pgfpathlineto{\pgfqpoint{2.815275in}{1.402490in}}%
\pgfpathlineto{\pgfqpoint{2.850246in}{1.407409in}}%
\pgfpathlineto{\pgfqpoint{2.885217in}{1.379212in}}%
\pgfpathlineto{\pgfqpoint{2.920187in}{1.343014in}}%
\pgfpathlineto{\pgfqpoint{2.955158in}{1.370725in}}%
\pgfpathlineto{\pgfqpoint{2.990129in}{1.313782in}}%
\pgfpathlineto{\pgfqpoint{3.025099in}{1.453188in}}%
\pgfpathlineto{\pgfqpoint{3.060070in}{1.413762in}}%
\pgfpathlineto{\pgfqpoint{3.095041in}{1.497335in}}%
\pgfpathlineto{\pgfqpoint{3.130011in}{1.514064in}}%
\pgfpathlineto{\pgfqpoint{3.164982in}{1.518845in}}%
\pgfpathlineto{\pgfqpoint{3.199953in}{1.540126in}}%
\pgfpathlineto{\pgfqpoint{3.234923in}{1.524766in}}%
\pgfpathlineto{\pgfqpoint{3.269894in}{1.537580in}}%
\pgfpathlineto{\pgfqpoint{3.304865in}{1.535825in}}%
\pgfpathlineto{\pgfqpoint{3.339835in}{1.488403in}}%
\pgfpathlineto{\pgfqpoint{3.374806in}{1.481168in}}%
\pgfpathlineto{\pgfqpoint{3.409777in}{1.427262in}}%
\pgfpathlineto{\pgfqpoint{3.444747in}{1.360560in}}%
\pgfpathlineto{\pgfqpoint{3.479718in}{1.235971in}}%
\pgfpathlineto{\pgfqpoint{3.514689in}{1.082082in}}%
\pgfpathlineto{\pgfqpoint{3.549659in}{1.106046in}}%
\pgfpathlineto{\pgfqpoint{3.584630in}{0.622470in}}%
\pgfpathlineto{\pgfqpoint{3.619601in}{1.020945in}}%
\pgfpathlineto{\pgfqpoint{3.654571in}{1.041172in}}%
\pgfpathlineto{\pgfqpoint{3.689542in}{0.960240in}}%
\pgfpathlineto{\pgfqpoint{3.724513in}{1.162924in}}%
\pgfpathlineto{\pgfqpoint{3.759483in}{1.124666in}}%
\pgfpathlineto{\pgfqpoint{3.794454in}{1.140074in}}%
\pgfpathlineto{\pgfqpoint{3.829425in}{1.185952in}}%
\pgfpathlineto{\pgfqpoint{3.864395in}{1.128071in}}%
\pgfpathlineto{\pgfqpoint{3.899366in}{1.251644in}}%
\pgfpathlineto{\pgfqpoint{3.934336in}{1.213815in}}%
\pgfpathlineto{\pgfqpoint{3.969307in}{1.349898in}}%
\pgfpathlineto{\pgfqpoint{4.004278in}{1.278899in}}%
\pgfpathlineto{\pgfqpoint{4.039248in}{1.408864in}}%
\pgfpathlineto{\pgfqpoint{4.074219in}{1.341082in}}%
\pgfpathlineto{\pgfqpoint{4.109190in}{1.333899in}}%
\pgfpathlineto{\pgfqpoint{4.144160in}{1.319099in}}%
\pgfpathlineto{\pgfqpoint{4.152411in}{1.277107in}}%
\pgfpathlineto{\pgfqpoint{4.152411in}{1.277107in}}%
\pgfusepath{stroke}%
\end{pgfscope}%
\begin{pgfscope}%
\pgfsetrectcap%
\pgfsetmiterjoin%
\pgfsetlinewidth{0.803000pt}%
\definecolor{currentstroke}{rgb}{0.000000,0.000000,0.000000}%
\pgfsetstrokecolor{currentstroke}%
\pgfsetdash{}{0pt}%
\pgfpathmoveto{\pgfqpoint{0.643596in}{0.515000in}}%
\pgfpathlineto{\pgfqpoint{0.643596in}{3.040900in}}%
\pgfusepath{stroke}%
\end{pgfscope}%
\begin{pgfscope}%
\pgfsetrectcap%
\pgfsetmiterjoin%
\pgfsetlinewidth{0.803000pt}%
\definecolor{currentstroke}{rgb}{0.000000,0.000000,0.000000}%
\pgfsetstrokecolor{currentstroke}%
\pgfsetdash{}{0pt}%
\pgfpathmoveto{\pgfqpoint{4.142411in}{0.515000in}}%
\pgfpathlineto{\pgfqpoint{4.142411in}{3.040900in}}%
\pgfusepath{stroke}%
\end{pgfscope}%
\begin{pgfscope}%
\pgfsetrectcap%
\pgfsetmiterjoin%
\pgfsetlinewidth{0.803000pt}%
\definecolor{currentstroke}{rgb}{0.000000,0.000000,0.000000}%
\pgfsetstrokecolor{currentstroke}%
\pgfsetdash{}{0pt}%
\pgfpathmoveto{\pgfqpoint{0.643596in}{0.515000in}}%
\pgfpathlineto{\pgfqpoint{4.142411in}{0.515000in}}%
\pgfusepath{stroke}%
\end{pgfscope}%
\begin{pgfscope}%
\pgfsetrectcap%
\pgfsetmiterjoin%
\pgfsetlinewidth{0.803000pt}%
\definecolor{currentstroke}{rgb}{0.000000,0.000000,0.000000}%
\pgfsetstrokecolor{currentstroke}%
\pgfsetdash{}{0pt}%
\pgfpathmoveto{\pgfqpoint{0.643596in}{3.040900in}}%
\pgfpathlineto{\pgfqpoint{4.142411in}{3.040900in}}%
\pgfusepath{stroke}%
\end{pgfscope}%
\begin{pgfscope}%
\definecolor{textcolor}{rgb}{0.000000,0.000000,0.000000}%
\pgfsetstrokecolor{textcolor}%
\pgfsetfillcolor{textcolor}%
\pgftext[x=2.393003in,y=3.124233in,,base]{\color{textcolor}{\rmfamily\fontsize{12.000000}{14.400000}\selectfont\catcode`\^=\active\def^{\ifmmode\sp\else\^{}\fi}\catcode`\%=\active\def%{\%}PMA3-14LN+}}%
\end{pgfscope}%
\begin{pgfscope}%
\pgfsetbuttcap%
\pgfsetmiterjoin%
\definecolor{currentfill}{rgb}{1.000000,1.000000,1.000000}%
\pgfsetfillcolor{currentfill}%
\pgfsetlinewidth{1.003750pt}%
\definecolor{currentstroke}{rgb}{0.800000,0.800000,0.800000}%
\pgfsetstrokecolor{currentstroke}%
\pgfsetdash{}{0pt}%
\pgfpathmoveto{\pgfqpoint{0.740818in}{1.563505in}}%
\pgfpathlineto{\pgfqpoint{1.388099in}{1.563505in}}%
\pgfpathquadraticcurveto{\pgfqpoint{1.415877in}{1.563505in}}{\pgfqpoint{1.415877in}{1.591283in}}%
\pgfpathlineto{\pgfqpoint{1.415877in}{1.964616in}}%
\pgfpathquadraticcurveto{\pgfqpoint{1.415877in}{1.992394in}}{\pgfqpoint{1.388099in}{1.992394in}}%
\pgfpathlineto{\pgfqpoint{0.740818in}{1.992394in}}%
\pgfpathquadraticcurveto{\pgfqpoint{0.713040in}{1.992394in}}{\pgfqpoint{0.713040in}{1.964616in}}%
\pgfpathlineto{\pgfqpoint{0.713040in}{1.591283in}}%
\pgfpathquadraticcurveto{\pgfqpoint{0.713040in}{1.563505in}}{\pgfqpoint{0.740818in}{1.563505in}}%
\pgfpathlineto{\pgfqpoint{0.740818in}{1.563505in}}%
\pgfpathclose%
\pgfusepath{stroke,fill}%
\end{pgfscope}%
\begin{pgfscope}%
\pgfsetrectcap%
\pgfsetroundjoin%
\pgfsetlinewidth{1.505625pt}%
\definecolor{currentstroke}{rgb}{0.000000,0.000000,1.000000}%
\pgfsetstrokecolor{currentstroke}%
\pgfsetstrokeopacity{0.700000}%
\pgfsetdash{}{0pt}%
\pgfpathmoveto{\pgfqpoint{0.768596in}{1.888227in}}%
\pgfpathlineto{\pgfqpoint{0.907485in}{1.888227in}}%
\pgfpathlineto{\pgfqpoint{1.046374in}{1.888227in}}%
\pgfusepath{stroke}%
\end{pgfscope}%
\begin{pgfscope}%
\definecolor{textcolor}{rgb}{0.000000,0.000000,0.000000}%
\pgfsetstrokecolor{textcolor}%
\pgfsetfillcolor{textcolor}%
\pgftext[x=1.157485in,y=1.839616in,left,base]{\color{textcolor}{\rmfamily\fontsize{10.000000}{12.000000}\selectfont\catcode`\^=\active\def^{\ifmmode\sp\else\^{}\fi}\catcode`\%=\active\def%{\%}$S_{21}$}}%
\end{pgfscope}%
\begin{pgfscope}%
\pgfsetrectcap%
\pgfsetroundjoin%
\pgfsetlinewidth{1.505625pt}%
\definecolor{currentstroke}{rgb}{1.000000,0.647059,0.000000}%
\pgfsetstrokecolor{currentstroke}%
\pgfsetstrokeopacity{0.700000}%
\pgfsetdash{}{0pt}%
\pgfpathmoveto{\pgfqpoint{0.768596in}{1.694616in}}%
\pgfpathlineto{\pgfqpoint{0.907485in}{1.694616in}}%
\pgfpathlineto{\pgfqpoint{1.046374in}{1.694616in}}%
\pgfusepath{stroke}%
\end{pgfscope}%
\begin{pgfscope}%
\definecolor{textcolor}{rgb}{0.000000,0.000000,0.000000}%
\pgfsetstrokecolor{textcolor}%
\pgfsetfillcolor{textcolor}%
\pgftext[x=1.157485in,y=1.646005in,left,base]{\color{textcolor}{\rmfamily\fontsize{10.000000}{12.000000}\selectfont\catcode`\^=\active\def^{\ifmmode\sp\else\^{}\fi}\catcode`\%=\active\def%{\%}$S_{11}$}}%
\end{pgfscope}%
\end{pgfpicture}%
\makeatother%
\endgroup%

  \end{center}
  \caption{PMA3-14LN+ gain and noise figure.}\label{fig:pma3}
\end{figure}

\section{Hardware Installation Guide}\label{sec:hig}
\subsection{RFI Shack Components}
shack panel, switching, plc, n9040b, other stuff. PLC cat6 pinout, rs232 db9 pinout,
\subsection{RF-DFS Components}
motor controller, motors, serial lps
\subsection{RF-EMS Components}
switching, amplifiers, lps,

\section{Communications}
maybe put pinouts and connections here instead to keep them all together
\subsection{Spectrum Analyzer}
scpi commands, how it works, etc, explain not to issue commands while program is running
\subsection{Motor Controller}
aries command reference stuff here, what bits are used
\subsection{PLC}
opcodes

\section{Software User Guide}
copy everything from github wiki
\subsection{Getting Started}
software installation stuff, ni-visa, other requirements
\subsection{Using the Program}
\subsubsection{Header}
\subsubsection{Control Panel}
\subsubsection{Antenna Position}
\subsubsection{Spectrum Analyzer Controls}
\subsubsection{Terminal}


\begin{figure}[!ht]
    \begin{center}
        \begin{circuitikz}
            \draw(0, 0) node[server, scale=1.5, name=computer]{};
            \draw(computer.south) node[anchor=north]{};
            \ctikzset{bipoles/oscope/waveform=triangle}
            \ctikzset{bipoles/oscope/width=1.0}
            \draw(computer.east)  ++ (0.6, 0) 
            coordinate(ctreast)-- ++ (1, 0)
            node[oscopeshape, anchor=west](spec){N9040b};
            \draw(spec.in 1) node[ocirc, scale=0.7]{};
            \draw(spec.in 2) node[ocirc, scale=0.7]{};
            \draw(spec.east) -- ++ (1.8, 0)
            node[twoportshape, name=s1]{} -- ++ (0.8, 0)
            to[amp, boxed, invert] ++ (1.7, 0) -- ++ (0.8, 0)
            node[twoportshape, name=s2]{};
            \draw(s1.center) node[spdt, scale=0.6]{};
            \draw(s2.center) node[spdt, xscale=-1, scale=0.6]{};
            \node[draw, rectangle, dashed, fit=(s1) (s2), inner sep=10](box){};

            \draw(s2.right up) -- ++ (1.7, 0)
            node[dinantenna, xscale=-1](ems){\ctikzflipx{EMS}};
            \draw(s2.right down) -- ++ (2.5, 0)
            node[dinantenna](dfs){DFS};

            \draw([yshift=5pt]box.north)[<-] -- ++ (0, 0.4)
            node[qfpchip, anchor=south, external pins width=0, hide numbers, name=uc, solid, scale=0.8]{PLC};
            
            
            \path[name path = border1](uc.west) -- ++ (-5, 0);
            \path[name path = line1, overlay]([yshift=18pt]ctreast) -- +(45:5);
            \draw[name intersections={of=border1 and line1}] ([yshift=18pt]ctreast) -- (intersection-1) -- (uc.west);
            \path[name path=border2](intersection-1) -- ++ (0, -5);
            \path[name path = line2, overlay]([yshift=-18pt]ctreast) -- +(-45:5);
            \draw[name intersections={of=border2 and line2}] ([yshift=-18pt]ctreast) -- (intersection-1) coordinate(p1);
            
            \draw(p1) -- ++ (2, 0)
            node[qfpchip, anchor=west, external pins width=0, hide numbers, name=acr, scale=0.8, align=center]{Motor\\Ctrl.};
            \draw([yshift=-8pt]acr.east) -- ++ (3.5, 0)
            node[elmech, rotate=90, anchor=top](m2){};
            \draw(m2) [->, dashed] -| ([yshift=-10pt]dfs.center);
            \draw([yshift=8pt]acr.east) -- ++ (1.5, 0)
            node[elmech, rotate=90, anchor=top](m1){};
            %\draw(m1) [->] -- ++ (3, 0) -- ([yshift=-10pt, xshift=-10pt]dfs.center);
            \path[name path = border3](dfs.center) -- ++ (0, -2);
            \path[name path = line3, overlay](m1.bottom) -- ++ (5, 0);
            \draw[name intersections={of=border3 and line3}, dashed] (m1.bottom) -- (intersection-1)
            node[diamondpole]{};
            \draw(m1.center) node{Az};
            \draw(m2.center) node{El};
        \end{circuitikz}
    \caption{Block diagram 2.}\label{fig:sysblock}
    \end{center}
\end{figure}

\begin{figure}[!ht]
  \begin{center}
      \begin{circuitikz}
          % \draw[help lines, dashed] grid (13, 3);
          % \draw[help lines, dashed] grid (13, -3);

          \draw(0, 0) node[bnc](port1){N9040b};
          \draw(port1.hot) -- ++ (1, 0)
          node[rotary switch = 2 in 30 wiper 20, anchor=in](sw1){};
        
          \draw(sw1.out 1) -| ++ (1.5, 1.5)
          -- ++ (0.5, 0)
          to[twoport, name=tp1, fill=green!20] ++ (2, 0)
          to[amp, invert, name=emsref1] ++ (2, 0)
          to[amp, invert] ++ (2, 0) -- ++ (0.5, 0)
          node[antenna]{EMS};
          % node[bnc, xscale=-1](port3){\ctikzflipx{EMS}};
          
          \draw(sw1.out 2) -| ++ (1.5, -1.5)
          -- ++ (0.5, 0)
          to[twoport, name=tp2, fill=green!20] ++ (2, 0)
          to[amp, invert, name=dfsref1] ++ (2, 0)
          to[amp, invert] ++ (2, 0) -- ++ (0.5, 0)
          node[antenna]{DFS};
          % node[bnc, xscale=-1](port3){\ctikzflipx{DFS}};

          % Attenuator
          \draw (tp1.center)
          node[resistorshape, anchor = center, rotate = 90, scale = 0.8]{};
          \draw (tp1.north)
          node[anchor=south]{62 ft Heliax};
          \draw (tp2.center)
          node[resistorshape, anchor = center, rotate = 90, scale = 0.8]{};
          \draw (tp2.north)
          node[anchor=south]{100 ft Heliax};

          % Switch labels and arrows
          \draw(sw1.north) node[anchor=south](sw1lab){S1};
          \draw(sw1lab.north) [dashed, <-] -- ++ (0, 2.2) node[anchor=south]{PLC};

          % Amplifier labels
          \draw(emsref1.north) ++ (1, 0.2) node[anchor=south]{PMA3-14LN+};
          \draw(dfsref1.north) ++ (1, 0.1) node[anchor=south]{PMA3-14LN+};
      \end{circuitikz}
  \caption{Phase 2 Block Diagram}\label{fig:ph2ampblock}
  \end{center}
\end{figure}


\end{document}


